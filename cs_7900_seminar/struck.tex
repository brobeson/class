\section{Tracking by Detection \& Struck}

\begin{frame}
    \frametitle{Tracking can be treated as an object detection problem.}
    \begin{columns}[T]
        \begin{column}{0.5\textwidth}
            \begin{description}
                \item [Tracking by detection] Object detection in each frame.
                \item [Adaptive tracking by detection] Update the classifier online.
            \end{description}
        \end{column}
        \begin{column}{0.5\textwidth}
            \begin{itemize}
                \item The algorithm operates on frame \(f_t\), for \(t \in \{1, 2, ..., T\}\).
                \item The tracker estimates a bounding box position, \(\mathbf{p}\).
                \item Extract features \(\mathbf{x}_t^\mathbf{p}\) from patches within the bounding box.
                \item Train the classifier with \((\mathbf{x}, z)\), where \(z = \pm1\).
            \end{itemize}
        \end{column}
    \end{columns}
\end{frame}

\begin{frame}
    \frametitle{Structured output tracking with kernels \cite{6126251}}
    \begin{itemize}
        \item Don't learn a binary classifier. Learn a prediction function: \(f : \mathcal{X} \rightarrow \mathcal{Y}\).
        \item Labelled sample is \((\mathbf{x}, \mathbf{y})\).
        \item \(\mathbf{y}\) is a translation in image space.
        \item \(f\) is learned in a structured output SVM.
        \item Use a discriminant function:
            \begin{align}
                F &: \mathcal{X} \times \mathcal{Y} \rightarrow \mathbb{R} \\
                F\left(\mathbf{x}, \mathbf{y}\right) &= \sum_{i, \mathbf{\bar{y}}} \beta_i^\mathbf{\bar{y}}
                    \langle \mathbf{\Phi}\left(\mathbf{x}_i, \mathbf{\bar{y}}\right),
                    \mathbf{\Phi}\left(\mathbf{x}, \mathbf{y}\right) \rangle \\
                \label{eq:discriminant_function}
                F\left(\mathbf{x}, \mathbf{y}\right) &= \sum_{i, \mathbf{\bar{y}}} \beta_i^\mathbf{\bar{y}}
                    \langle \kappa\left(\mathbf{x}_i, \mathbf{\bar{y}}\right),
                    \kappa\left(\mathbf{x}, \mathbf{y}\right) \rangle \\
                \mathbf{y}_t &= \argmax\limits_{y \in \mathcal{Y}} F \left( \mathbf{x}_t^{\mathbf{p}_{t-1}}, \mathbf{y} \right)
            \end{align}
    \end{itemize}
\end{frame}
