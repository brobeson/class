\documentclass[fleqn]{article}
\usepackage[table]{xcolor}

\begin{document}
% top matter
\title{CS 6680 - Final Exam}
\author{Brendan Robeson}
\maketitle

\begin{description}
\item [1.1]
    \begin{math}p \oplus 3B =\end{math}

    \begin{tabular}{| c | c | c | c | c | c | c | c | c | c | c | c | c | c | c | c | c |}
        \hline
        & & & \cellcolor{gray} & \cellcolor{gray} & \cellcolor{gray} & \cellcolor{gray} & \cellcolor{gray} & \cellcolor{gray} & \cellcolor{gray} & & & & & & & \\ \hline
        & & & \cellcolor{gray} & \cellcolor{gray} & \cellcolor{gray} & \cellcolor{gray} & \cellcolor{gray} & \cellcolor{gray} & \cellcolor{gray} & & & & & & & \\ \hline
        & & & \cellcolor{gray} & \cellcolor{gray} & \cellcolor{gray} & \cellcolor{gray} & \cellcolor{gray} & \cellcolor{gray} & \cellcolor{gray} & & & & & & & \\ \hline
        & & & \cellcolor{gray} & \cellcolor{gray} & \cellcolor{gray} & \cellcolor{gray} & \cellcolor{gray} & \cellcolor{gray} & \cellcolor{gray} & & & & & & & \\ \hline
        & & & \cellcolor{gray} & \cellcolor{gray} & \cellcolor{gray} & \cellcolor{gray} & \cellcolor{gray} & \cellcolor{gray} & \cellcolor{gray} & & & & & & & \\ \hline
        & & & \cellcolor{gray} & \cellcolor{gray} & \cellcolor{gray} & \cellcolor{gray} & \cellcolor{gray} & \cellcolor{gray} & \cellcolor{gray} & & & & & & & \\ \hline
        & & & \cellcolor{gray} & \cellcolor{gray} & \cellcolor{gray} & \cellcolor{gray} & \cellcolor{gray} & \cellcolor{gray} & \cellcolor{gray} & & & & & & & \\ \hline
        & & & & & & & & & & & & & & & & \\ \hline
        & & & & & & & & & & & & & & & & \\ \hline
        & & & & & & & & & & & & & & & & \\ \hline
        & & & & & & & & & & & & & & & & \\ \hline
        & & & & & & & & & & & & & & & & \\ \hline
    \end{tabular}

    \begin{math}(p \oplus 3B) \cap A =\end{math}

    \begin{tabular}{| c | c | c | c | c | c | c | c | c | c | c | c | c | c | c | c | c |}
        \hline
        & & & & & & & & & & & & & & & & \\ \hline
        & & & & & & & & & & & & & & & & \\ \hline
        & & & \cellcolor{gray} & \cellcolor{gray} & \cellcolor{gray} & \cellcolor{gray} & \cellcolor{gray} & & & & & & & & & \\ \hline
        & & & \cellcolor{gray} & \cellcolor{gray} & \cellcolor{gray} & \cellcolor{gray} & \cellcolor{gray} & & & & & & & & & \\ \hline
        & & & \cellcolor{gray} & \cellcolor{gray} & \cellcolor{gray} & \cellcolor{gray} & \cellcolor{gray} & & & & & & & & & \\ \hline
        & & & & & \cellcolor{gray} & \cellcolor{gray} & \cellcolor{gray} & \cellcolor{gray} & \cellcolor{gray} & & & & & & & \\ \hline
        & & & \cellcolor{gray} & \cellcolor{gray} & \cellcolor{gray} & \cellcolor{gray} & \cellcolor{gray} & & & & & & & & & \\ \hline
        & & & & & & & & & & & & & & & & \\ \hline
        & & & & & & & & & & & & & & & & \\ \hline
        & & & & & & & & & & & & & & & & \\ \hline
        & & & & & & & & & & & & & & & & \\ \hline
        & & & & & & & & & & & & & & & & \\ \hline
    \end{tabular}

\item [1.2]
    The geodetic distance is 5.
    \begin{displaymath}
        p1_x - p_x = 12 - 7 = 5
    \end{displaymath}

\item [1.3] \begin{equation} A \oplus B \end{equation}
B is a disk structuring element of radius 8, with its origin at the center of
the disk.
\item [2.1.a]
\begin{displaymath}
T = \{ 3, 2, 2, 5, 2, 2 ,7, 7, 7, 6, 7, 8, 6, 7, 3, 4, 5 \}
\end{displaymath}
\item [2.2.a]
    The minimums are the region of 1's, region of 2's and region of 3's. They
        are highlighted below. All other regions have an adjacent region of
        lower intensity; these three regions all have adjacent regions which are
        only of higher intensity.

    \begin{tabular}{| c | c | c | c | c | c | c | c |}
        \hline
        8 & 8 & 7 & 8 & 9 & 9 & 9 & 9 \\ \hline
        8 & 8 & \cellcolor{green} 3 & \cellcolor{green} 3 & 9 & 9 & 9 & 9 \\ \hline
        8 & 8 & 9 & 9 & 9 & 9 & 9 & 9 \\ \hline
        8 & 7 & 6 & 8 & 9 & 9 & 9 & 9 \\ \hline
        7 & 6 & 6 & 7 & 7 & 9 & \cellcolor{green} 1 & \cellcolor{green} 1 \\ \hline
        7 & \cellcolor{green} 2 & \cellcolor{green} 2 & 7 & 6 & 5 & \cellcolor{green} 1 & \cellcolor{green} 1 \\ \hline
        7 & 6 & 6 & 7 & 11 & 12 & 5 & 5 \\ \hline
        7 & 7 & 6 & 7 & 11 & 11 & 7 & 5 \\
        \hline
    \end{tabular}
\end{description}

\end{document}
