\documentclass[fleqn]{article}
\usepackage{amsmath}
\usepackage[table]{xcolor}

\begin{document}
% top matter
\title{CS 6680 - Final Exam}
\author{Brendan Robeson}
\maketitle

\begin{description}

% problem 1.1 {{{
\item [1.1]
    \begin{math}p \oplus 3B =\end{math}

    \begin{tabular}{| c | c | c | c | c | c | c | c | c | c | c | c | c | c | c | c | c |}
        \hline
        & & & \cellcolor{gray} & \cellcolor{gray} & \cellcolor{gray} & \cellcolor{gray} & \cellcolor{gray} & \cellcolor{gray} & \cellcolor{gray} & & & & & & & \\ \hline
        & & & \cellcolor{gray} & \cellcolor{gray} & \cellcolor{gray} & \cellcolor{gray} & \cellcolor{gray} & \cellcolor{gray} & \cellcolor{gray} & & & & & & & \\ \hline
        & & & \cellcolor{gray} & \cellcolor{gray} & \cellcolor{gray} & \cellcolor{gray} & \cellcolor{gray} & \cellcolor{gray} & \cellcolor{gray} & & & & & & & \\ \hline
        & & & \cellcolor{gray} & \cellcolor{gray} & \cellcolor{gray} & \cellcolor{gray} & \cellcolor{gray} & \cellcolor{gray} & \cellcolor{gray} & & & & & & & \\ \hline
        & & & \cellcolor{gray} & \cellcolor{gray} & \cellcolor{gray} & \cellcolor{gray} & \cellcolor{gray} & \cellcolor{gray} & \cellcolor{gray} & & & & & & & \\ \hline
        & & & \cellcolor{gray} & \cellcolor{gray} & \cellcolor{gray} & \cellcolor{gray} & \cellcolor{gray} & \cellcolor{gray} & \cellcolor{gray} & & & & & & & \\ \hline
        & & & \cellcolor{gray} & \cellcolor{gray} & \cellcolor{gray} & \cellcolor{gray} & \cellcolor{gray} & \cellcolor{gray} & \cellcolor{gray} & & & & & & & \\ \hline
        & & & & & & & & & & & & & & & & \\ \hline
        & & & & & & & & & & & & & & & & \\ \hline
        & & & & & & & & & & & & & & & & \\ \hline
        & & & & & & & & & & & & & & & & \\ \hline
        & & & & & & & & & & & & & & & & \\ \hline
    \end{tabular}

    \begin{math}(p \oplus 3B) \cap A =\end{math}

    \begin{tabular}{| c | c | c | c | c | c | c | c | c | c | c | c | c | c | c | c | c |}
        \hline
        & & & & & & & & & & & & & & & & \\ \hline
        & & & & & & & & & & & & & & & & \\ \hline
        & & & \cellcolor{gray} & \cellcolor{gray} & \cellcolor{gray} & \cellcolor{gray} & \cellcolor{gray} & & & & & & & & & \\ \hline
        & & & \cellcolor{gray} & \cellcolor{gray} & \cellcolor{gray} & \cellcolor{gray} & \cellcolor{gray} & & & & & & & & & \\ \hline
        & & & \cellcolor{gray} & \cellcolor{gray} & \cellcolor{gray} & \cellcolor{gray} & \cellcolor{gray} & & & & & & & & & \\ \hline
        & & & & & \cellcolor{gray} & \cellcolor{gray} & \cellcolor{gray} & \cellcolor{gray} & \cellcolor{gray} & & & & & & & \\ \hline
        & & & \cellcolor{gray} & \cellcolor{gray} & \cellcolor{gray} & \cellcolor{gray} & \cellcolor{gray} & & & & & & & & & \\ \hline
        & & & & & & & & & & & & & & & & \\ \hline
        & & & & & & & & & & & & & & & & \\ \hline
        & & & & & & & & & & & & & & & & \\ \hline
        & & & & & & & & & & & & & & & & \\ \hline
        & & & & & & & & & & & & & & & & \\ \hline
    \end{tabular}
% }}}

% problem 1.2 {{{
\item [1.2]
    The geodetic distance is 5.
    \begin{displaymath}
        p1_x - p_x = 12 - 7 = 5
    \end{displaymath}
% }}}

% problem 1.3 {{{
\item [1.3] \begin{equation} A \oplus B \end{equation}
B is a disk structuring element of radius 8, with its origin at the center of
the disk.

\item [2.1.a]
\begin{displaymath}
T = \{ 3, 2, 2, 5, 2, 2 ,7, 7, 7, 6, 7, 8, 6, 7, 3, 4, 5 \}
\end{displaymath}
% }}}

% problem 2.2.a {{{
\item [2.2.a]
    The minimums are the region of 1's, region of 2's and region of 3's. They
        are highlighted below. All other regions have an adjacent region of
        lower intensity; these three regions all have adjacent regions which are
        only of higher intensity.

    \begin{tabular}{| c | c | c | c | c | c | c | c |}
        \hline
        8 & 8 & 7 & 8 & 9 & 9 & 9 & 9 \\ \hline
        8 & 8 & \cellcolor{green} 3 & \cellcolor{green} 3 & 9 & 9 & 9 & 9 \\ \hline
        8 & 8 & 9 & 9 & 9 & 9 & 9 & 9 \\ \hline
        8 & 7 & 6 & 8 & 9 & 9 & 9 & 9 \\ \hline
        7 & 6 & 6 & 7 & 7 & 9 & \cellcolor{green} 1 & \cellcolor{green} 1 \\ \hline
        7 & \cellcolor{green} 2 & \cellcolor{green} 2 & 7 & 6 & 5 & \cellcolor{green} 1 & \cellcolor{green} 1 \\ \hline
        7 & 6 & 6 & 7 & 11 & 12 & 5 & 5 \\ \hline
        7 & 7 & 6 & 7 & 11 & 11 & 7 & 5 \\
        \hline
    \end{tabular}
% }}}

% problem 2.2.b {{{
\item [2.2.b]
    Flooding the catchment basins to n = 8 (using \begin{math}g(s,t) <
    n\end{math}):
    \begin{align*}
        T[2] &= \{ (7,5) (8,5) (7,6) (8,6) \} \\
        T[3] &= \{ (7,5) (8,5) (7,6) (8,6) T[2] \} \\
        T[4] &= \{ (3,2) (3,3) T[3] \} \\
        T[5] &= T[4] \\
        T[6] &= \{ (6,6) (7,7) (8,7) (8,8) T[5] \} \\
        T[7] &= \{ (3,4) (2,5) (3,5) (5,6) (2,7) (3,7) (3,8) T[6] \} \\
        T[8] &= \{ (3,1) (2,4) (1,5) (4,5) (5,5) (1,6) (4,6) (1,7) (4,7) (1,8) (2,8) (4,8) (7,8) T[7] \}
    \end{align*}

    \begin{tabular}{| c | c | c | c | c | c | c | c |}
        \hline
        8 & 8 & \cellcolor{green} 7 & 8 & 9 & 9 & 9 & 9 \\ \hline
        8 & 8 & \cellcolor{green} 3 & \cellcolor{green} 3 & 9 & 9 & 9 & 9 \\ \hline
        8 & 8 & 9 & 9 & 9 & 9 & 9 & 9 \\ \hline
        8 & \cellcolor{green} 7 & \cellcolor{green} 6 & 8 & 9 & 9 & 9 & 9 \\ \hline
        \cellcolor{green} 7 & \cellcolor{green} 6 & \cellcolor{green} 6 & \cellcolor{green} 7 & \cellcolor{green} 7 & 9 & \cellcolor{green} 1 & \cellcolor{green} 1 \\ \hline
        \cellcolor{green} 7 & \cellcolor{green} 2 & \cellcolor{green} 2 & \cellcolor{green} 7 & \cellcolor{green} 6 & \cellcolor{green} 5 & \cellcolor{green} 1 & \cellcolor{green} 1 \\ \hline
        \cellcolor{green} 7 & \cellcolor{green} 6 & \cellcolor{green} 6 & \cellcolor{green} 7 & 11 & 12 & \cellcolor{green} 5 & \cellcolor{green} 5 \\ \hline
        \cellcolor{green} 7 & \cellcolor{green} 7 & \cellcolor{green} 6 & \cellcolor{green} 7 & 11 & 11 & \cellcolor{green} 7 & \cellcolor{green} 5 \\
        \hline
    \end{tabular}

    \begin{align*}
        C \left (M_1 \right ) \cap T[7] &= \{ (7,5) (8,5) (4,6) (5,6) (6,6) (7,6) (7,7) (8,7) (8,8) \} \\
        C \left (M_2 \right ) \cap T[7] &= \{ (3,4) (2,5) (3,5) (2,6) (3,6) (2,7) (3,7) (3,8) \} \\
        C \left (M_3 \right ) \cap T[7] &= \{ (3,2) (4,2) \}
    \end{align*}

    Gray pixels indicate the watershed line as it exists at this point:

    \begin{tabular}{| c | c | c | c | c | c | c | c |}
        \hline
        8 & 8 & \cellcolor{red}7 & 8 & 9 & 9 & 9 & 9 \\ \hline
        8 & 8 & \cellcolor{red} 3 & \cellcolor{red} 3 & 9 & 9 & 9 & 9 \\ \hline
        8 & 8 & 9 & 9 & 9 & 9 & 9 & 9 \\ \hline
        8 & \cellcolor{blue}7 & \cellcolor{blue} 6 & 8 & 9 & 9 & 9 & 9 \\ \hline
        \cellcolor{blue}7 & \cellcolor{blue} 6 & \cellcolor{blue} 6 & \cellcolor{lightgray} 7 & \cellcolor{green} 7 & 9 & \cellcolor{green} 1 & \cellcolor{green} 1 \\ \hline
        \cellcolor{blue}7 & \cellcolor{blue} 2 & \cellcolor{blue} 2 & \cellcolor{lightgray} 7 & \cellcolor{green} 6 & \cellcolor{green} 5 & \cellcolor{green} 1 & \cellcolor{green} 1 \\ \hline
        \cellcolor{blue}7 & \cellcolor{blue} 6 & \cellcolor{blue} 6 & \cellcolor{lightgray} 7 & 11 & 12 & \cellcolor{green} 5 & \cellcolor{green} 5 \\ \hline
        \cellcolor{blue}7 & \cellcolor{blue}7 & \cellcolor{blue} 6 & \cellcolor{blue}7 & 11 & 11 & \cellcolor{green} 7 & \cellcolor{green} 5 \\
        \hline
    \end{tabular}

    The final set of watershed lines:

    \begin{tabular}{| c | c | c | c | c | c | c | c |}
        \hline
        8 & 8 & 7 & 8 & 9 & 9 & 9 & 9 \\ \hline
        8 & 8 & 3 & 3 & 9 & 9 & 9 & 9 \\ \hline
        \cellcolor{lightgray} 8 & \cellcolor{lightgray} 8 & \cellcolor{lightgray} 9 & \cellcolor{lightgray} 9 & 9 & 9 & 9 & 9 \\ \hline
        \cellcolor{lightgray} 8 & 7 & 6 & \cellcolor{lightgray} 8 & \cellcolor{lightgray} 9 & \cellcolor{lightgray} 9 & \cellcolor{lightgray} 9 & \cellcolor{lightgray} 9 \\ \hline
        7 & 6 & 6 & \cellcolor{lightgray} 7 & 7 & 9 & 1 & 1 \\ \hline
        7 & 2 & 2 & \cellcolor{lightgray} 7 & 6 & 5 & 1 & 1 \\ \hline
        7 & 6 & 6 & \cellcolor{lightgray} 7 & \cellcolor{lightgray} 11 & 12 & 5 & 5 \\ \hline
        7 & 7 & 6 & 7 & \cellcolor{lightgray} 11 & 11 & 7 & 5 \\
        \hline
    \end{tabular}
% }}}

% problem 3.1 {{{
\item [3.1]
    Initial state:
    \begin{align*}
        X &= \{ (0,0), (0,1), (5,4), (5,5), (4,5), (1,0) \} \\
        K &= 2 \\
        z_1(1) &= (0,0) \\
        z_2(1) &= (0,1)
    \end{align*}
    Iteration 1:
    \begin{align*}
        S_1(1) &= \{ (0,0), (1,0) \} \\
        S_2(1) &= \{ (0,1), (5,4), (5,5), (4,5) \} \\
        z_1(2) &= \frac{1}{2} \sum_{x=S_1(1)}{x} = (0.5, 0) \\
        z_2(2) &= \frac{1}{4} \sum_{x=S_2(1)}{x} = (3.5, 3.75) \\
        z_1(1) &\neq z_1(2), z_2(1) \neq z_2(2)
    \end{align*}
    Iteration 2:
    \begin{align*}
        S_1(2) &= \{ (0,0), (1,0), (0,1) \} \\
        S_2(2) &= \{ (5,4), (5,5), (4,5) \} \\
        z_1(3) &= \frac{1}{3} \sum_{x=S_1(1)}{x} = (0.3333, 0.3333) \\
        z_2(3) &= \frac{1}{3} \sum_{x=S_2(1)}{x} = (4.6667, 4.6667) \\
        z_1(2) &\neq z_1(3), z_2(2) \neq z_2(3)
    \end{align*}
    Iteration 3:
    \begin{align*}
        S_1(3) &= \{ (0,0), (1,0), (0,1) \} \\
        S_2(3) &= \{ (5,4), (5,5), (4,5) \} \\
        z_1(4) &= \frac{1}{3} \sum_{x=S_1(1)}{x} = (0.3333, 0.3333) \\
        z_2(4) &= \frac{1}{3} \sum_{x=S_2(1)}{x} = (4.6667, 4.6667) \\
        z_1(3) &= z_1(4), z_2(3) = z_2(4)
    \end{align*}
    The operation is complete with iteration 3. The clusters are
    \begin{math}S_1(3), S_2(3)\end{math}
% }}}

\end{description}

\end{document}
