\documentclass{IEEEtran}
\usepackage{amsmath}
\title{CS 7650 Assignment 1}
\author{Brendan Robeson}
\date{2017 Fall}

\begin{document}
\maketitle

% Problem 1a {{{
\section{Problem 1a}
From the book, on page 23:
\[ P(error|x) = min[P(\omega_1|x),P(\omega_2|x)] \]
Show that this can be replaced with
\[ P(error|x) = 2P(\omega_1|x)P(\omega_2|x) \]
in
\[ P(error) = \int_{-\infty}^\infty P(error|x)p(x)dx \]
to get an upper bound on \(P(error|x)\).
\subsection{Solution}
Assume \( P(\omega_1|x) \le P(\omega_2|x) \). \\
Then \( P(error|x) = P(\omega_1|x) \). \\
Given \( P(\omega_1|x) + P(\omega_2|x) = 1 \). \\
Then
\begin{align*}
    2P(\omega_2|x) &\ge 1 \\
    2P(\omega_1|x)P(\omega_2|x) &\ge P(\omega_1|x) \\
    2P(\omega_1|x)P(\omega_2|x) & \ge P(error|x)
\end{align*}
Thus, \(2P(\omega_1|x)P(\omega_2|x)\) is an upper bound for \(P(error|x)\).
% }}}

\newpage

% Problem 1b {{{
\section{Problem 1b}
Show that for \(P(error|x) = \alpha P(\omega_1|x)P(\omega_2|x)\), for \(\alpha < 2\), an upper bound
is not guaranteed.
\subsection{Solution}
Assume
\begin{equation}\label{eq:alpha}
    \alpha P(\omega_1|x)P(\omega_2|x) \ge P(\omega_1|x)
\end{equation}
Given
\begin{equation}
    \frac{1}{2} \le P(\omega_2|x) \le \frac{1}{\alpha}
\end{equation}
Let \(\alpha = 1.7\). \\
Then \(\frac{1}{\alpha} = 0.588\). \\
Let \(P(\omega_2|x) = 0.52\). \\
Then \(P(\omega_1|x) = 0.48\). \\
Plug into \ref{eq:alpha}:
\begin{align*}
    \alpha P(\omega_1|x)P(\omega_2|x) &\ge P(\omega_1|x) \\
    1.7 \times 0.48 \times 0.52 &\ge 0.48 \\
    0.42 &\ge 0.48
\end{align*}
This is a contradiction.
% }}}

\newpage

% Problem 2a {{{
\section{Problem 2a}
\subsection{Solution}
\begin{align*}
    p(x|\omega_1) &= k e^{-\frac{|x-a_i|}{b_i}} \\
    1 &= k e^{-\frac{|x-a_i|}{b_i}} \\
    &= \int\limits_{-\infty}^{a_i} e^{\frac{x-a_i}{b_i}}dx + \int\limits_{a_i}^{\infty} e^{-\frac{x-a_i}{b_i}}dx \\
    &= e^{\frac{a_i-a_i}{b_i}} - e^{\frac{\infty-a_i}{b_i}} + e^{-\frac{\infty-a_i}{b_i}} - e^{-\frac{a_i-a_i}{b_i}}
\end{align*}
% }}}

\newpage

% Problem 2b {{{
\section{Problem 2b}
\subsection{Solution}
Given
\begin{equation}
    p(x|\omega_i) = \frac{1}{2b_i}e^{\frac{-|x-a_i|}{b_i}}
\end{equation}
Then
\begin{align*}
    \frac{p(x|\omega_1)}{px|\omega_2)} &= \frac{\frac{1}{2b_1}e^{\frac{-|x-a_1|}{b_1}} }
                                               {\frac{1}{2b_2}e^{\frac{-|x-a_2|}{b_2}} } \\
                                       &= \frac{\frac{1}{b_1}e^{\frac{-|x-a_1|}{b_1}} }
                                               {\frac{1}{b_2}e^{\frac{-|x-a_2|}{b_2}} } \\
                                       &= \frac{b_1}{b_2}\frac{e^{\frac{-|x-a_1|}{b_1}} }
                                                              {e^{\frac{-|x-a_2|}{b_2}} } \\
                                       &= \frac{b_1}{b_2}e^{\frac{-|x-a_1|}{b_1} - \frac{-|x-a_2|}{b_2}} \\
                                       &= \frac{b_1}{b_2}e^{\frac{|x-a_2|}{b_2} - \frac{|x-a_1|}{b_1}}
\end{align*}
% }}}

\newpage

% Problem 4a {{{
\section{Problem 4a}
\begin{align*}
    R = & \int_{R_1}\left[\lambda_{11}P(\omega_1)p(x|\omega_1) + \lambda_{12}P(\omega_2)p(x|\omega_2)\right]dx + \\
        & \int_{R_2}\left[\lambda_{21}P(\omega_1)p(x|\omega_1) + \lambda_{22}P(\omega_2)p(x|\omega_2)\right]dx \\
    R = & \int_{R_1}\lambda_{11}P(\omega_1)p(x|\omega_1)dx + \int_{R_1}\lambda_{12}P(\omega_2)p(x|\omega_2)dx + \\
        & \int_{R_2}\lambda_{21}P(\omega_1)p(x|\omega_1)dx + \int_{R_2}\lambda_{22}P(\omega_2)p(x|\omega_2)dx \\
    R = & \lambda_{11}P(\omega_1)\int_{R_1}p(x|\omega_1)dx + \lambda_{12}P(\omega_2)\int_{R_1}p(x|\omega_2)dx + \\
        & \lambda_{21}P(\omega_1)\int_{R_2}p(x|\omega_1)dx + \lambda_{22}P(\omega_2)\int_{R_2}p(x|\omega_2)dx
\end{align*}

Given
\begin{align}
    \int_{R_2}p(x|\omega_2)dx &= 1 - \int_{R_1}p(x|\omega_1)dx \\
    P(\omega_2) &= 1 - P(\omega_1)
\end{align}

\subsubsection{Lanbda 11 term}
\begin{align*}
    &= \lambda_{11}P(\omega_1)\int_{R_1}p(x|\omega_1)dx \\
    &= \lambda_{11}P(\omega_1)\left[1 - \int_{R_2}p(x|\omega_2)dx \right] \\
    &= \lambda_{11}\left[P(\omega_1) - P(\omega_1)\int_{R_2}p(x|\omega_2)dx \right]
\end{align*}

\subsubsection{Lanbda 12 term}
\begin{align*}
    &= \lambda_{12}P(\omega_2)\int_{R_1}p(x|\omega_2)dx \\
    &= \lambda_{12}\left[1 - P(\omega_2)\right]\int_{R_1}p(x|\omega_2)dx \\
    &= \lambda_{12}\left[\int_{R_1}p(x|\omega_2)dx - P(\omega_2)\int_{R_1}p(x|\omega_2)dx \right]
\end{align*}

\subsubsection{Lanbda 22 term}
\begin{align*}
    &= \lambda_{22}P(\omega_2)\int_{R_2}p(x|\omega_2)dx \\
    &= \lambda_{22}\left[1 - P(\omega_1)\right] \left[1 - \int_{R_1}p(x|\omega_1)dx \right] \\
    &= \lambda_{22}\left[1 - P(\omega_1) - \int_{R_1}p(x|\omega_1)dx + P(\omega_1)\int_{R_1}p(x|\omega_1)dx \right]
\end{align*}

\subsubsection{Combined}
\begin{align*}
    R = & \lambda_{11}\left[P(\omega_1) - P(\omega_1)\int_{R_2}p(x|\omega_2)dx \right] + \\
        & \lambda_{12}\left[\int_{R_1}p(x|\omega_2)dx - P(\omega_2)\int_{R_1}p(x|\omega_2)dx \right] + \\
        & \lambda_{21}P(\omega_1)\int_{R_2}p(x|\omega_1)dx + \\
        & \lambda_{22}\left[1 - P(\omega_1) - \int_{R_1}p(x|\omega_1)dx + P(\omega_1)\int_{R_1}p(x|\omega_1)dx \right] \\
    R = & \lambda_{11}P(\omega_1) - \lambda_{11}P(\omega_1)\int_{R_2}p(x|\omega_2)dx + \\
        & \lambda_{12}\int_{R_1}p(x|\omega_2)dx - \lambda_{12}P(\omega_2)\int_{R_1}p(x|\omega_2)dx + \\
        & \lambda_{21}P(\omega_1)\int_{R_2}p(x|\omega_1)dx + \\
        & \lambda_{22} - \lambda_{22}P(\omega_1) - \lambda_{22}\int_{R_1}p(x|\omega_1)dx + \lambda_{22}P(\omega_1)\int_{R_1}p(x|\omega_1)dx \\
    R = & \lambda_{11}P(\omega_1) - \lambda_{11}P(\omega_1)\int_{R_2}p(x|\omega_2)dx + \\
        & \lambda_{21}P(\omega_1)\int_{R_2}p(x|\omega_1)dx - \lambda_{22}P(\omega_1) + \\
        & \lambda_{22}P(\omega_1)\int_{R_1}p(x|\omega_1)dx \\
        & \lambda_{12}\int_{R_1}p(x|\omega_2)dx - \lambda_{12}P(\omega_2)\int_{R_1}p(x|\omega_2)dx + \\
        & \lambda_{22} - \lambda_{22}\int_{R_1}p(x|\omega_1)dx \\
    R = & \left[\lambda_{11} - \lambda_{11}\int_{R_2}p(x|\omega_2)dx + \lambda_{21}\int_{R_2}p(x|\omega_1)dx - \lambda_{22} + \lambda_{22}\int_{R_1}p(x|\omega_1)dx \right] \\
        & P(\omega_1) +\\
        & \lambda_{12}\int_{R_1}p(x|\omega_2)dx - \lambda_{12}P(\omega_2)\int_{R_1}p(x|\omega_2)dx + \\
        & \lambda_{22} - \lambda_{22}\int_{R_1}p(x|\omega_1)dx \\
    R = & \left[\left(\lambda_{11} - \lambda_{22}\right) - \lambda_{11}\int_{R_2}p(x|\omega_2)dx + \lambda_{21}\int_{R_2}p(x|\omega_1)dx + \lambda_{22}\int_{R_1}p(x|\omega_1)dx \right] \\
        & P(\omega_1) +\\
        & \lambda_{12}\int_{R_1}p(x|\omega_2)dx - \lambda_{12}P(\omega_2)\int_{R_1}p(x|\omega_2)dx + \\
        & \lambda_{22} - \lambda_{22}\int_{R_1}p(x|\omega_1)dx
\end{align*}
% }}}

\newpage

% Problem 12a {{{
\section{Problem 12a}
\subsection{Solution}
If \(P(\omega_i|\textbf{x}) = P(\omega_j|\textbf{x})\), for all i and j, then \(P(\omega_{max}|\textbf{x}) = \frac{1}{c}\).
If any \(P(\omega_i|\textbf{x} < \frac{1}{c}\), then some \(P(\omega_j|\textbf{x}) > \frac{1}{c}, i \ne j\).
% }}}

\newpage

% Problem 12b {{{
\section{Problem 12b}
\subsection{Solution}
\begin{align*}
    P(correct) &= \int P(\omega_{max}|\textbf{x})p(\textbf{x})d\textbf{x} \\
    P(error) &= 1 - P(correct) \\
    &= 1 - \int P(\omega_{max}|\textbf{x})p(\textbf{x})d\textbf{x}
\end{align*}
% }}}

\newpage

% Problem 12c {{{
\section{Problem 12c}
\subsection{Solution}
\begin{align*}
    P(error) &= 1 - \int P(\omega_{max}|\textbf{x})p(\textbf{x})d\textbf{x} \\
    &\le 1 - \int \frac{1}{c} p(\textbf{x})d\textbf{x} \\
    &\le 1 - \frac{1}{c} \int p(\textbf{x})d\textbf{x} \\
    &\le 1 - \frac{1}{c} 1 \\
    &\le 1 - \frac{1}{c}
\end{align*}
% }}}

\end{document}

