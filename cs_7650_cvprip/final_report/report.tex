\documentclass{IEEEtran}

%\newcommand{\Vector}{\textbf{#1}}

\begin{document}
\title{Fuzzy Structured Output Tracking with Kernels}
\author{Brendan Robeson}
\date{\today}
\maketitle

\begin{abstract}
    Tracking an object in a video or image stream has many practical applications, from enhancing a
    sports video broadcast, to military surveillance and reconnaissance. A common approach to the
    problem of object tracking is tracking-by-detection. This involves the use of pattern
    recognition techniques to classify samples of the current frame as the target object, or not the
    target object. Object occlusion is frequently an area which causes tracking-by-detection
    algorithms to fail. In this paper, I build on a fairly robust tracking algorithm, introducing
    fuzzy logic to the employed SVM.
\end{abstract}

\section{Introduction}

\section{F-Struck}
\subsection{Struck}
Struck is an adaptive tracking algorithm developed by Hare, et. al. \cite{6126251}. For frame N, the
algorithm samples radially around the object location in frame N-1. A structured output SVM is used
to predict the location of the object in frame N. Once the object is found, the SVM is updated; thus
it learns as the object is tracked in the sequence.

A traditional SVM takes, as training input, data of the form \((\textbf{x}, y)\), where
\(\textbf{x}\) is a feature vector, and \(y \in \{-1, +1\}\) is a class label. The training input of
a structured output SVM takes the form \((\textbf{x}, \textbf{y})\). As in a traditional SVM,
\(\textbf{x}\) is a feature vector. In this case, however, \(\textbf{y} \in need\_fancy\_font(Y)\) is a prediction
function. In Struck, the prediction function is a transformation from the object location, to the
sample location.

\subsection{Fuzzy SVM}

\subsection{Fuzzy Struck}

\section{Experiments}

\section{Conclusion}

%\bibliographystyle{IEEEtran}
%\bibliography{references}

\end{document}
