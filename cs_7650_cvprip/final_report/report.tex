\documentclass{IEEEtran}
\usepackage{color}
\usepackage{amsmath}

\renewcommand{\vec}[1]{\mathbf{#1}}
\newcommand{\half}{\frac{1}{2}}
\newcommand{\loss}{\Delta\left(\vec{y}, \vec{y}_i\right)}

\begin{document}
\title{Fuzzy Structured Output Tracking with Kernels}
\author{Brendan Robeson}
\date{\today}
\maketitle

\begin{abstract} %start_fold_1
    Tracking an object in a video or image stream has many practical applications, from enhancing a
    sports video broadcast, to military surveillance and reconnaissance. A common approach to the
    problem of object tracking is tracking-by-detection. This involves the use of pattern
    recognition techniques to classify samples of the current frame as the target object, or not the
    target object. Object occlusion is frequently an area which causes tracking-by-detection
    algorithms to fail. In this paper, I build on a fairly robust tracking algorithm, introducing
    fuzzy logic to the employed SVM.
\end{abstract}

\section{Introduction} %start_fold_1
\subsection{Fuzzy SVM} %start_fold_2
Fuzzy SVM reformulates the SVM to incorporate fuzzy membership data for the classes \cite{991432}. The
traditional SVM formula
\begin{align}
    \min_\vec{w} & \half \vec{w} \cdot \vec{w} + C\sum_{i=1}^n\xi_i \\
    \text{s.t. } & \forall i \xi_i \ge 0 \nonumber \\
                 & \forall i y_i \left(\vec{w} \cdot \vec{x}_i + b\right) \ge 1 - \xi_i \nonumber
\end{align}

becomes
\begin{align}
    \min_\vec{w} & \half \vec{w} \cdot \vec{w} + C\sum_{i=1}^ns_i\xi_i \\
    \text{s.t. } & \forall i \xi_i \ge 0 \nonumber \\
                 & \forall i y_i \left(\vec{w} \cdot \vec{x}_i + b\right) \ge 1 - \xi_i \nonumber
\end{align}
where \(s_i\) is the fuzzy membership.

The dual problem is then
\begin{align}
    \max_\alpha & \sum_{i=1}^n \alpha_i - \half \sum_{i=1}^n \sum_{j=1}^n \alpha_i \alpha_j
    y_i y_j K\left(\vec{x}_i, \vec{x}_j\right) \\
    \text{s.t. } & \sum_{i=1}^n y_i \alpha_i = 0 \nonumber \\
                 & 0 \le \alpha_i \le s_iC \nonumber
\end{align}

Fuzzy membership is calculated by a function \(s_i = f(\cdot)\). The exact function depends on the
classification problem to be solved, but should satisfy \(0 \le \theta \le f(\cdot) \le 1\).
\(\theta\) is a predefined lower bound, also dependent on the classification problem.

\subsection{Struck} %start_fold_2
Struck is an adaptive tracking algorithm developed by Hare, et. al. \cite{6126251}. For frame N, the
algorithm samples radially around the object location in frame N-1. A structured output SVM is used
to predict the location of the object in frame N. Once the object is found, the SVM is updated; thus
it learns as the object is tracked in the sequence.

A traditional SVM takes, as training input, data of the form \((\vec{x}, y)\), where
\(\vec{x}\) is a feature vector, and \(y \in \{-1, +1\}\) is a class label. The training input of
a structured output SVM takes the form \((\vec{x}, \vec{y})\). As in a traditional SVM,
\(\vec{x}\) is a feature vector. In this case, however, \(\vec{y} \in \mathcal{Y}\) is a prediction
function. In Struck, the prediction function is a transformation from the object location, to the
sample location.

\begin{align}
    \min_\vec{w} & \half \vec{w} \cdot \vec{w} + C \sum_{i=1}^n \xi_i \\
    \text{s.t. } & \forall i : \xi_i \ge 0 \nonumber \\
    & \forall i, \forall \vec{y} \ne \vec{y}_i : \vec{w} \cdot \delta \vec{\Phi}_i(\vec{y})
    \ge \delta(\vec{y}_i, \vec{y}) - \xi_i \nonumber
\end{align}

Optimizing in terms of \(\alpha\)

\begin{align}
    \max_\alpha & \sum_{i,\vec{y}\ne\vec{y}_i} \loss
    \alpha_i^{\vec{y}} - \half \sum_{\substack{i,\vec{y}\ne\vec{y}_i \\ j,
    \vec{y}\ne\vec{y}_j}} \alpha_i^\vec{y} \alpha_j^{\bar{\vec{y}}} \delta
    \vec{\Phi}_i(\vec{y}) \cdot \delta \vec{\Phi}(\bar{\vec{y}}) \\
    % subject to...
    \text{s.t. } & \forall i, \forall \vec{y} \ne \vec{y}_i : \alpha_i^\vec{y} \ge 0 \nonumber \\
    & \forall i : \sum_{\vec{y} \ne \vec{y}_i} \alpha_i^\vec{y} \le C \nonumber
\end{align}

Reparametrizing in terms of \(\beta\)

\begin{align}
    \max_\beta & -\sum_{i,\vec{y}} \loss \beta_i^\vec{y} - \half
    \sum_{i,\vec{y},j,\vec{\bar{y}}} \beta_i^\vec{y} \beta_j^\vec{\bar{y}}
    \vec{\Phi}\left(\vec{x}_i, \vec{y}\right) \cdot \vec{\Phi}\left(\vec{x}_j,
    \vec{\bar{y}}\right) \\
    % subject to...
    \text{s.t. } & \forall i, \forall \vec{y} : \beta_i^\vec{y} \le
    \begin{cases}
        0 & \vec{y} \ne \vec{y}_i \\
        C & \vec{y} = \vec{y}_i
    \end{cases} \nonumber \\
    & \forall i : \sum_\vec{y} \beta_i^\vec{y} = 0 \nonumber
\end{align}

Discriminant function

\begin{displaymath}
    F(\vec{x}, \vec{y}) = \sum_{i,\vec{\bar{y}}} \beta_i^\vec{\bar{y}}
    \vec{\Phi}\left(\vec{x}_i, \vec{\bar{y}}\right) \cdot \vec{\Phi}\left(\vec{x},
    \vec{y}\right)
\end{displaymath}

Gradient
\begin{align}
    g_i\left(\vec{y}\right) &= - \loss - \sum_{j,\vec{\bar{y}}} \beta_j^\vec{\bar{y}}
    \vec{\Phi}\left(\vec{x}_i, \vec{y}\right) \cdot \vec{\Phi}\left(\vec{x}_j,
    \vec{y}\right) \\
    &= - \loss - F\left(\vec{x}_i, \vec{y}\right) \nonumber
\end{align}

\section{F-Struck} %start_fold_1

\subsection{Fuzzy Structured Output SVM}

\subsection{Fuzzy Membership} %start_fold_2
For F-Struck, the fuzzy membership function is defined as a function of bounding box overlap. In
this context, a sample bounding box with high overlap will contain more of the target; low overlap
will contain less of the target.

\begin{displaymath}
    s_i = f\left(b, b_i\right) = \frac{A(b \cap b_i)}{A(b \cup b_i)}
\end{displaymath}

where \(b\) defines the target bounding box, and \(b_i\) defines the sample bounding box, and
\(A(x)\) is the area of \(x\). If the two bounding boxes do not intersect, then \(A\left(b \cap
b_i\right) = 0\), and thus \(s_i = 0\). If the bounding boxes coincide exactly,
\(A\left(b \cap b_i\right) = A\left(b \cup b_i\right)\), and \(s_i = 1\).

\section{Experiments}

\section{Notes} %start_fold_1
\begin{align*}
    \min_\vec{w} & \half \vec{w} \cdot \vec{w} + C \sum_{i=1}^n \textcolor{red}{s_i} \xi_i \\
    \text{s.t. } & \textcolor{red}{s_i} \xi_i \ge 0 \\
    & \vec{w} \cdot \delta \vec{\Phi}_i(y) \ge \loss - \textcolor{red}{s_i} \xi_i
\end{align*}

\begin{align*}
    \max_\alpha & \sum_i \loss \alpha_i^y - \half \sum_{i,j} \alpha_i^y
    \alpha_j^{\bar{y}} \delta \vec{\Phi}_i(y) \cdot \delta \vec{\Phi}(\bar{y}) \\
    \text{s.t. } & \forall y \ne y_i : \alpha_i^y \ge 0 \\
    & \sum_{y \ne y_i} \alpha_i^y \le C
\end{align*}

\newpage

\subsection{LaGrange} %start_fold_2
\begin{align}
    f &= \half \vec{w} \cdot \vec{w} + C \sum_{i=1}^n \textcolor{red}{s_i} \xi_i \\
    g_1 &= \xi_i \ge 0 \\
    g_2 &= \vec{w} \cdot \delta \vec{\Phi}\left(y_i\right) + \textcolor{red}{s_i} \xi_i - \loss \ge 0 \\
    &= \vec{w} \cdot \left(\vec{\Phi}\left(y_i\right) - \vec{\Phi}(y)\right) + \textcolor{red}{s_i} \xi_i
    - \loss \ge 0 \nonumber
\end{align}
\begin{align}
    J &= f - \alpha g_1 - \beta g_2 \\
    J &= \half \vec{w} \cdot \vec{w} + C \sum_{i=1}^n \textcolor{red}{s_i} \xi_i -
        \sum_i \alpha_i \xi_i \nonumber \\ & -
        \sum_i \beta_i \left[\vec{w} \cdot \delta \vec{\Phi}\left(y_i\right) + \textcolor{red}{s_i} \xi_i
        - \loss\right] \nonumber \\
    J &= \half \vec{w} \cdot \vec{w} + C \sum_{i=1}^n \textcolor{red}{s_i} \xi_i -
        \sum_i \alpha_i \xi_i \nonumber \\ & -
        \sum_i \left[ \beta_i \vec{w} \cdot \delta \vec{\Phi}\left(y_i\right) + \beta_i
        \textcolor{red}{s_i} \xi_i - \beta_i \loss\right] \nonumber \nonumber
\end{align}

% derivative with respect to w
\begin{align}
    \frac{\partial J}{\partial \vec{w}} = 0 &= \vec{w} - \sum_i \beta_i \delta \vec{\Phi}\left(y_i\right) \\
    \vec{w} &= \sum_i \beta_i \delta \vec{\Phi}\left(y_i\right) \nonumber
\end{align}

% derivative with respect to ξ
\begin{align}
    \frac{\partial J}{\partial \xi_i} = 0 &= C \textcolor{red}{s_i} - \alpha_i - \beta_i \textcolor{red}{s_i}
\end{align}

\section{Conclusion} %start_fold_1

\bibliographystyle{IEEEtran} %start_fold_1
\bibliography{references}

\end{document}
