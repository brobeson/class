\Group{Struck}

\begin{frame}
    \frametitle{Structured output tracking with kernels \cite{6126251}}
    \begin{itemize}
        \item Don't learn a binary classifier. Learn a prediction function: \(f : \mathcal{X} \rightarrow \mathcal{Y}\).
        \item Labelled sample is \((\mathbf{x}_i, \mathbf{y}_i)\).
        \item \(\mathbf{y}_i\) is a translation in image space.
        \item \(f\) is learned in a structured output SVM.
        \item Use a discriminant function \(F : \mathcal{X} \times \mathcal{Y} \to \mathbb{R}\)
            \struckDiscriminant
    \end{itemize}
\end{frame}

\begin{frame}
    \frametitle{Struck SVM}
    \struckEquation
\end{frame}

\begin{frame}
    \frametitle{Struck SVM}
    \begin{itemize}
        \item \(\delta \vec{\Phi}_i \left( \vec{y} \right) = \vec{\Phi} \left( \vec{x}_i, \vec{y}_i \right) - \vec{\Phi} \left( \vec{x}_i, \vec{y} \right)\)
        \item \(\Delta\) is a loss function
            \begin{itemize}
                \item \(\Delta \left(\vec{y}, \bar{\vec{y}} \right) = 0\) iff \(\vec{y} =
                    \bar{\vec{y}}\)
                \item \(\Delta \left(\vec{y}, \bar{\vec{y}} \right) \to 0\) as \(\vec{y} \to
                    \bar{\vec{y}}\)
            \end{itemize}
        \item For Struck, \(\loss = 1 - s_{\vec{p}_t}^o \left( \vec{y}, \vec{\bar{y}}
            \right) \) (1 - bounding box overlap)
    \end{itemize}
\end{frame}

\begin{frame}
    \frametitle{Optimizing Struck SVM}
    Optimizing according to LaRank \cite{Bordes:2007:SMS:1273496.1273508}
    \struckDualAlpha
\end{frame}

\begin{frame}
    \frametitle{Reparametrizing the Optimization}
    Reparametrizing in terms of \(\beta\) \cite{Bordes:2007:SMS:1273496.1273508}
    \[ \beta_i^\vec{y} = \begin{cases}
        -\alpha_i^\vec{y} & \vec{y} \ne \vec{y}_i \\
        \sum_{\vec{\bar{y}} \ne \vec{y}_i} \alpha_i^\vec{\bar{y}} & \vec{y} = \vec{y}_i
    \end{cases} \]
    \struckDualBeta
\end{frame}

\begin{frame}
    \frametitle{Struck terminology}
    \begin{description}
        \item [Support vector] \( \left( \vec{x}_i, \vec{y} \right) \) for which \(\beta_i^\vec{y}
            \ne 0 \)
        \item [Support pattern] \(\vec{x}_i\) in at least one support vector
        \item [Positive support vector] Support vector \( \left( \vec{x}_i, \vec{y}_i \right) \)
            with \( \beta_i^{\vec{y}_i} > 0 \)
        \item [Negative support vector] Support vector \( \left( \vec{x}_i, \vec{y} \right), \vec{y}
            \ne \vec{y}_i \) with \( \beta_i^\vec{y} < 0 \)
    \end{description}
\end{frame}

\begin{frame}
    \frametitle{Sequential Minimal Optimization Step}
    \begin{columns}[T]
        \begin{column}{0.5\textwidth}
            \struckSmo
        \end{column}
        \begin{column}{0.5\textwidth}
            \begin{align*}
                K_{++} &= \vec{\Phi} (\vec{x}_i, \vec{x}_+) \cdot \vec{\Phi} (\vec{x}_i, \vec{y}_+) \\
                K_{--} &= \vec{\Phi} (\vec{x}_i, \vec{x}_-) \cdot \vec{\Phi} (\vec{x}_i, \vec{y}_-) \\
                K_{+-} &= \vec{\Phi} (\vec{x}_i, \vec{x}_-) \cdot \vec{\Phi} (\vec{x}_i, \vec{y}_-) \\
                \delta(\vec{y}_+, \vec{y}_-) &= \begin{cases}
                    0 & \vec{y}_+ \ne \vec{y}_- \\
                    1 & \vec{y}_+ = \vec{y}_-
                \end{cases}
            \end{align*}
            Adapted from \cite{sequential-minimal-optimization-a-fast-algorithm-for-training-support-vector-machines}
        \end{column}
    \end{columns}
\end{frame}

%\begin{frame}
%    \frametitle{What is the new discriminant function?}
%    \struckNewDiscriminant
%\end{frame}
%
%\begin{frame}
%    \frametitle{Calculating the Gradients}
%    \struckGradients
%\end{frame}

\begin{frame}
    \frametitle{Support vector budget}
    \begin{itemize}
        \item With each frame, new support vectors are added to the SVM.
        \item Unbounded number of support vectors increases computation time, and can lead to
            overfitting.
        \item Struck imposes a limit on the number of support vectors.
        \item If the bound is to be exceeded, support vectors are removed, and coefficients are
            updated.
        \item Remove vectors which result in the smallest \(\|\Delta \vec{w}\|^2\)
        \item According to \cite{6126251}: \struckWeightChange
    \end{itemize}
\end{frame}
