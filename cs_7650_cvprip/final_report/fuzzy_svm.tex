\section{Fuzzy SVM}

\begin{frame}
    \frametitle{What is a fuzzy SVM? \cite{991432}}
    \begin{itemize}
        \item Add fuzzy class membership to SVMs
        \item Training data becomes \(\left(\vec{x}_i, y_i, s_i\right)\)
        \item \(s_i\) is the fuzzy membership of feature vector \(\vec{x}_i\) in class \(y_i\).
        \item \(s_i \in [0, 1]\)
    \end{itemize}
\end{frame}

\begin{frame}
    \frametitle{Converting traditional SVM to fuzzy SVM}
    \begin{columns}[T]
        \begin{column}{0.5\textwidth}
            Traditional SVM
            \svmEquation
        \end{column}
        \begin{column}{0.5\textwidth}
            Fuzzy SVM
            \fuzzySvmEquation
        \end{column}
    \end{columns}
\end{frame}

\begin{frame}
    \frametitle{Converting traditional SVM to fuzzy SVM}
    \begin{columns}[T]
        \begin{column}{0.5\textwidth}
            Traditional SVM Dual Optimization
            \traditionalDual
            For \(\vec{x}_i\) with \(0 < \alpha_i < C\), \(\vec{x}_i\) is a support vector.

            If \(\alpha_i = C\), \(\vec{x}_i\) is erroneous.
        \end{column}
        \begin{column}{0.5\textwidth}
            Fuzzy SVM Dual Optimization
            \fuzzyDual
            As \(s_i \to 0\), the bound on \(\alpha_i\) tightens.
        \end{column}
    \end{columns}
\end{frame}

\begin{frame}
    \frametitle{Defining fuzzy membership}
    \begin{itemize}
        \item Fuzzy membership is dependent on the classification problem
        \item Define a lower bound, \(\theta \in [0, 1]\)
        \item Define \(f(\cdot) \in [\theta, 1]\) based on a property of the data
        \item \(s_i = f(\cdot)\)
    \end{itemize}
\end{frame}
