\documentclass[11pt]{article}
\usepackage[margin=1in]{geometry}
\usepackage{enumitem}
\usepackage{algorithm2e}
\usepackage{hyperref}
\hypersetup{colorlinks=true}

\begin{document}
%\noindent Brendan Robeson \hfill CS 7680 - Assignment 5 \hfill \today
\noindent Brendan Robeson

\noindent CS 7680 - Assignment 4

\noindent \today

\medskip

\begin{description}[leftmargin=0in]
    \item [Source] A. Jazayeri et al. "Vehicle Detection and Tracking in Car Video Based on Motion
        Model", \emph{IEEE Transactions on Intelligent Transportation Systems}, vol. 12, no. 2,
        June, 2011, pp. 583-595. Cited by 177.

    \item [URL]
        \url{http://ieeexplore.ieee.org/abstract/document/5723749/}.

    \item [Problem] The authors want to track vehicles in video recorded with a camera mounted in a
        vehicle, typically a dash-cam. This problem has applications with police pursuit, driver
        notifications, and self-driving cars.

    \item [Assumptions] The authors make two assumptions:
        \begin{enumerate}
            \item The camera will not turn away from, nor fall back from, the target vehicles.
            \item The camera must remain in motion.
        \end{enumerate}

    \item [Data Sets] The authors do not describe how they acquired their data, nor if they are
        available for download.

    \item [Algorithm Overview]
        The data modeled by the HMM is referred to as traces. A trace is essentially a description
        of horizontal motion. The second assumption, that the camera remains in motion, is used for
        extracting traces. Traces rely on motion in the video frame, so a stationary background will
        have no motion.

        %In my opinion, the authors confused HMM states and observations in this paper. What they
        %call states, strike me as observations; and vice versa. That said, I'll continue the summary
        %using the authors' descriptions.

        The authors did not train their HMM; they explicitly state this on page 592. Their testing
        process appears to have been simply to run their algorithm on some data, and observe the
        results.

        The HMM involves two states: car and background. That is, a trace can represent a car or the
        background. The HMM observations are the trace's position in the image, \emph{x}, and its
        horizontal velocity, \emph{v}.

        The transition array is given by Equation (20). The authors give a general rationale for the
        array values, but not describe how they arrived at the specific values used. For example,
        the explain that it unlikely for a car trace to transition to a background trace, but do not
        describe how the derived the specific probability of 0.8 for the car-to-car transition. The
        initial probabilities were empirically determined to be 0.7 for a trace to be a car, and 0.3
        for it to be background. No more detail than that is provided for their determination. The
        observation array does not appear to be used. Instead, two look up tables are generated, one
        for each state. The look up tables provide probability density functions, mapping position
        and velocity values to probability for the table's state. These tables are generated
        offline, using equations described below.

        Equation (13) gives the probability of specific values for position and velocity, given a
        background trace, for straight motion. Equation (14) extends the PDF to include slight
        heading changes by the vehicle. This uses the first assumption described above. Very small
        variations in heading allow the authors to simply the inclusion of this term in Equation
        (14). Equations (16) and (17) are the equivalent of (13) and (14), except for car traces. 

        Equations (21) and (22) describe the posterior probabilities that a trace represents a car
        or background, respectively. Equation (21) calculates two probabilities: 1) the probability
        that the trace was background in the previous frame, and will transition to car this frame,
        scaled by the observation likelihood using the car PDF table; and 2) the probability that
        the trace was a car in the previous frame, and will transition to a car this frame, again
        scaled by the same likelihood value. The equation uses the maximum of these two
        probabilities. Equation (22) is analogous, but using transition to background and likelihood
        from the background PDF. If the result of (21) is greater than (22), the trace is classified
        as a car. Otherwise, it is considered background. The trace is only finalized after it has
        been tracked for a minimum number of frames. Traces lost before that threshold are
        considered noise, and discarded.

    \item [Experiments] The authors did not compare their technique to others. The authors provided
        several graphs of example trace probabilities, as well as video frames with bounding boxes
        around vehicles. The authors did not state any performance metrics. The only performance
        data provided is a confusion matrix for cars and background.

    \item [Contributions] The authors claim their method avoids dependency on threshold parameters,
        thus is more stable. They also claim their algorithm is effective, and not computationally
        intensive. This indicates the algorithm is suitable for an embedded application in vehicle
        hardware.

    \item [Shortcomings] The technique requires "mild driving conditions." Constant speed in a
        straight line is best; abrupt changes in speed, or curves, can create problems for the
        tracker. Also, vehicle far away are difficult to detect due to their small size. Finally,
        the tracker requires significant number of video frames to be certain it's tracking a true
        positive.

    \item [Self Evaluation] I think the lack of comparison to other tracking methods is a weakness
        for this paper. We have no idea how the HMM used compares to other techniques. They also did
        not provide any comparison to ground truth. I found the technique to be unintuitive, but
        interesting after I was able to comprehend it.

    \item [Improvements] The authors made assumptions regarding the motion of the camera vehicle,
        which allowed them to simplify their equations. I would look into solving the problem
        without those simplifications. If that could be accomplished, the tracker would be more
        robust under more realistic driving conditions.

    \item [Applications] I think it would be interesting to apply this technique to airborne video.
        The math would need to be modified a bit; the assumed motion is different for a car mounted
        camera than for a helicopter, airplane, or drone camera.

    \item [Packages]

        Probabilistic Modeling Toolkit for Matlab (\url{https://github.com/probml/pmtk3})

        Shogun (\url{http://www.shogun-toolbox.org/})

\end{description}

 \end{document}
