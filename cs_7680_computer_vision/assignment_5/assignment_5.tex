\documentclass[11pt]{article}
\usepackage[margin=1in]{geometry}
\usepackage{enumitem}
\usepackage{algorithm2e}
\usepackage{hyperref}
\hypersetup{colorlinks=true}

\begin{document}
\noindent Brendan Robeson \hfill CS 7680 - Assignment 5 \hfill \today
%\noindent Brendan Robeson
%
%\noindent CS 7680 - Assignment 4
%
%\noindent \today

\medskip

\begin{description}[leftmargin=0in]
    \item [Source] A. Jazayeri et al. "Vehicle Detection and Tracking in Car Video Based on Motion
        Model", \emph{IEEE Transactions on Intelligent Transportation Systems}, vol. 12, no. 2,
        June, 2011, pp. 583-595. Cited by 177.

    \item [URL]
        \url{http://ieeexplore.ieee.org/abstract/document/5723749/}.

    \item [Problem] The authors want to track vehicles in video recorded with a camera mounted in a
        vehicle, typically a dash-cam. This problem has applications with police pursuit, driver
        notifications, and self-driving cars.

    \item [Assumptions] The authors make two assumptions:
        \begin{enumerate}
            \item The camera will not turn away from, nor fall back from, the target vehicles.
            \item The camera must remain in motion.
        \end{enumerate}

    \item [Data Sets] The authors do not describe how they acquired their data, nor if they are
        available for download.

    \item [Algorithm Overview]
        The data modeled by the HMM is referred to as traces. A trace is essentially a description
        of horizontal motion.

        In my opinion, the authors confused HMM states and observations in this paper. What they
        call states, strike me as observations; and vice versa. That said, I'll continue the summary
        using the authors' descriptions.

        The HMM involves two states: car and background. That is, a trace can represent a car or the
        background. The HMM observations are the trace's position in the image, \emph{x}, and its
        horizontal velocity, \emph{v}.

        Equation (14) gives the probability that the background is observed.

    \item [Experiments] The authors did not compare their technique to others. The authors provided
        several graphs of example trace probabilities, as well as video frames with bounding boxes
        around vehicles. The authors did not state any performance metrics. The only performance
        data provided is a confusion matrix for cars and background.

    \item [Contributions] The authors claim their method avoids dependency on threshold parameters,
        thus is more stable. They also claim their algorithm is effective, and not computationally
        intensive. This indicates the algorithm is suitable for an embedded application in vehicle
        hardware.

    \item [Shortcomings] The technique requires "mild driving conditions." Constant speed in a
        straight line is best; abrupt changes in speed, or curves, can create problems for the
        tracker. Also, vehicle far away are difficult to detect due to their small size. Finally,
        the tracker requires significant number of video frames to be certain it's tracking a true
        positive.

    \item [Self Evaluation]

    \item [Improvements] The authors made assumptions regarding the motion of the camera vehicle,
        which allowed them to simplify their equations. I would look into solving the problem
        without those simplifications. If that could be accomplished, the tracker would be more
        robust under more realistic driving conditions.

    \item [Applications] I think it would be interesting to apply this technique to airborne video.
        The math would need to be modified a bit; the assumed motion is different for a car mounted
        camera than for a helicopter, airplane, or drone camera.

    \item [Packages]

        Probabilistic Modeling Toolkit for Matlab (\url{https://github.com/probml/pmtk3})

        Shogun (\url{http://www.shogun-toolbox.org/})

\end{description}

 \end{document}
