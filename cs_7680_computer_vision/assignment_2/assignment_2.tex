\documentclass[11pt]{article}
\usepackage[margin=1in]{geometry}

\begin{document}
\noindent Brendan Robeson

\noindent CS 7680 - Assignment 2

\noindent \today

\medskip

\begin{enumerate}
    \item Zhao and Zhang \cite{s111009573} has 47 citations according to Google
        Scholar.

    \item The paper is available at
        http://www.mdpi.com/1424-8220/11/10/9573/htm.

    \item The authors want the computer to recognize facial expressions in
        images. Facial expressions are a primary mode of non-verbal
        communication between people. Thus automating facial recognition has
        interesting applications for various fields.

    \item The authors did not state any assumptions.

    \item The authors used the JAFFE database (http://www.kasrl.org/jaffe.html)
        and the Cohn-Kanade database
        (http://www.pitt.edu/~emotion/ck-spread.htm).

    \item The authors dub their algorithm kernel discriminant isomap (KDIsomap).
        The goal of KDIsomap is to maximize interclass scatter, while minimizing
        intraclass scatter, as per LDA. The authors begin with kernal isomap
        (KIsomap). The crux of KDIsomap is modify the Euclidean distance used in
        KIsomap.

        Input for the algorithm is the image data and the corresponding class
        label. The image data point is a vector of length D. The class label is
        one of the facial expressions of interest.

        The first step is to implicitly map the input data to a higher dimensional
        space. A nonlinear function, \(\phi\), performs this mapping. A kernel
        function is used to avoid actually performing the math in higher
        dimensions, as shown with equation (6). For their experiments, the
        authors used a Gaussian kernel function; the equation is not numbered,
        but is provided in the text on page 9581.

        Second, a neighborhood graph is constructed for the high dimensional
        data. The data points form the graph vertices. The edges connect a point
        to its nearest neighbors, and are weighted with the kernel discriminant
        distance between two vertices. Equation (8) provides the formula for the
        kernel discriminant distance as a function of the data points' classes,
        and the data points' Euclidean distance. The Euclidean distance, in the
        high dimensional space, is given by Equation (7). Equation (8) is how
        the algorithm maintains the scatter constraints outlined above.

        The remaining steps are taken from KIsomap. The third step of KDIsomap
        is to calculate a matrix of geodesic distances between all the data
        points. The authors suggest using Dijkstra's algorithm

        The fourth step is to calculate the matrix \(K(D^2)\) using equation
        (3).

        Next, calculate a Mercer kernel matrix \(K^*\) from equation (4). From
        \(K^*\) calculate the \emph{d} most important eigenvectors and
        corresponding eigenvalues.

        The final step is to calculate the embedded coordinates. Equation (5)
        describes this process. These are the output of the algorithm.

    \item KDIsomap was compared to PCA, LDA, KPCA, kernel linear discriminant
        analysis (KLDA), and kernel isomap (KIsomap).

        Quantitatively, the authors compared each algorithm's recognition
        accuracy vs. embedded dimension in a line graph. The authors also
        provide a table showing each algorithm's peak accuracy with the
        corresponding embedded dimension. In addition, a confusion matrix is
        provided for KDIsomap for seven classes of facial expressions. These
        data are provided separately for the JAFFE and Cohn-Kanade databases.
        Finally, the authors provided a table of computational and memory
        complexity for each algorithm.

        There are no qualitative figures directly related to KDIsomap.

    \item KDIsomap improves facial expression classification over KIsomap. It
        does so by minimizing scatter among data within a class, and maximizing
        scatter among data between classes.

    \item The authors did not mention any shortcomings in the paper. They did
        discuss integrating KDIsomap with boosted LBP and an SVM classifier.
        They also hypothesize that incorporating time information could lead to
        improved accuracy.

    \item This is a somewhat novel approach to the problem. Their experiments
        show significant accuracy improvement over KIsomap, but only minor
        improvement over the other methods. Computational complexity is equal to
        that of KPCA and KLDA, and worse than PCA and LDA.

    \item

    \item This technique is applicable to any feature extraction task, provided
        the computational costs are acceptable.

    \item
\end{enumerate}

\bibliographystyle{IEEEtran}
\bibliography{references}
\end{document}
