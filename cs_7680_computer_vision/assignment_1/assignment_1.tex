\documentclass[11pt]{article}
\usepackage[margin=1in]{geometry}

\begin{document}
\title{CS 7680 - Assignment 1}
\author{Brendan Robeson}
\maketitle

Tian et al. \cite{5688239} has 55 citations according to Google Scholar, and is
available at \\ http://ieeexplore.ieee.org.dist.lib.usu.edu/document/5688239/.
The primary goal of the paper is to improve segmentation of brain tissue in MR
imagery. They accomplish this by improving the EM algorithm frequently used in
MR imagery segmentation. Segmenting MR imagery allows doctors to better analyze
the imagery and focus on the tissue of concern. Manual segmentation is time
consuming; subjective; and difficult, if not impossible, to reproduce. These
issues make manual segmentation impractical for large neurological studies. The
authors point out that the traditional EM algorithm tends to over fit the
training data, and can become focused on local optima. These drawbacks limit the
accuracy of EM. The authors also state that EM is deterministic, implying this
is also a drawback. However, this property seems to me to be an advantage.

The authors make standard assumptions about their data as required by GMM. Brain
MR imagery is composed of volume elements, or voxels. The authors assume that
each voxel represents a single type of tissue; that is, a voxel will not contain
a bit of tissue type 1 and a bit of tissue type 2. For their experiments, the
authors assumed three Gaussian models for the EM and variational EM (VEM)
algorithms.

Two data sets were used for verification and experimentation. The first
consisted of 20 low resolution brain MR images. The second set was 17 high
resolution brain MR images. Both data sets were obtained from The Internet Brain
Segmentation Repository (IBSR). The repository's URL is
http://www.cma.mgh.harvard.edu/ibsr/index.html. The authors do not identify
which specific data sets they retrieved from the repository.

\[TODO - Algorithmic Overview\]

The authors compared their GA-VEM algorithm to standard Expectation Maximization
(EM), Variational EM (VEM), and to Genetic Algorithm Expectation Maximization
(GA-EM). In addition, they compared their segmentation to that of the software
packages Statistical Parametric Mapping (SPM) and FSL.

\bibliographystyle{IEEEtran}
\bibliography{references}
\end{document}
