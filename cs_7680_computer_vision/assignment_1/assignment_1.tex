\documentclass[11pt]{article}
\usepackage[margin=1in]{geometry}

\begin{document}
\noindent Brendan Robeson

\noindent CS 7680 - Assignment 1

\noindent \today

\medskip

Tian et al. \cite{5688239} has 55 citations according to Google Scholar, and is
available at \\ http://ieeexplore.ieee.org.dist.lib.usu.edu/document/5688239/.
The primary goal of the paper is to improve segmentation of brain tissue in MR
imagery. They accomplish this by improving the EM algorithm frequently used in
MR imagery segmentation. Segmenting MR imagery allows doctors to better analyze
the imagery and focus on the tissue of concern. Manual segmentation is time
consuming; subjective; and difficult, if not impossible, to reproduce. These
issues make manual segmentation impractical for large neurological studies. The
authors point out that the traditional EM algorithm tends to over fit the
training data, and can become focused on local optima. These drawbacks limit the
accuracy of EM. The authors also state that EM is deterministic, implying this
is also a drawback. However, this property seems to me to be an advantage.

The authors make standard assumptions about their data as required by GMM. Brain
MR imagery is composed of volume elements, or voxels. The authors assume that
each voxel represents a single type of tissue; that is, a voxel will not contain
a bit of tissue type 1 and a bit of tissue type 2. For their experiments, the
authors assumed three Gaussian models for the EM and variational EM (VEM)
algorithms.

Two data sets were used for verification and experimentation. The first
consisted of 20 low resolution brain MR images. The second set was 17 high
resolution brain MR images. Both data sets were obtained from The Internet Brain
Segmentation Repository (IBSR). The repository's URL is
http://www.cma.mgh.harvard.edu/ibsr/index.html, though attempts to visit
repository timed out. The authors do not identify which specific data sets they
retrieved from the repository.

The goal of the genetic algorithm (GA) initialization process is to determine
the best set of GMM parameters, \(\Theta\). The process starts with a population
of parameter sets; the authors use 500 sets. Each initial set has parameters
with random, though valid, values. GA is an iterative process. For each
iteration, the fitness value for each parameter set is calculated. The set with
the highest fitness is retained for the next iteration. All other parameter sets
are combined to create new parameter sets. The authors refer the reader to two
references for the combination operation: BLX-0.5 operator and a permutation
operator. The next iteration then operates on the new parameter set. After a
predetermined number of iterations, the parameter set with highest fitness score
is used to initialize the VEM hyperparameter set, \(\Psi\). Equation (18) describes the
iterative process for calculating \(\alpha^0\). \(\beta^0\) is initialized to
0.1; this value was found empirically by the authors. \(\nu^0\) is initialized
to the number of observed data points. \(m^0\) and \(W^0\) are initialized to
the Gaussian component's mean and the inverse of its covariance matrix (the
precision matrix).

The VEM algorithm is an iterative process, involving two basic steps for each
iteration. The first step is maximize \(L(q;\Psi)\) with respect to
\(q(\Theta)\). During this step, \(\Psi\) remains constant. Equation (15) is
calculated for each distribution. The maximum result is kept for the second
step, maximizing \(L(q;\Psi)\) with respect to \(q(Z)\). This is done by
calculating the maximum value of equation (16). Equations (13) and (14) show how
the two results are combined. This process iterates until the difference in
\(\Psi\) from one iteration to the next is below an acceptable threshold.
Equations (17) describe how to update the parameters in \(\Psi\).

The authors compared their GA-VEM algorithm to standard Expectation Maximization
(EM), Variational EM (VEM), and to Genetic Algorithm EM (GA-EM). The GA-EM
algorithm tested was obtained from the MIXTUREGA software package, available at
\\ http://www.cs.tut.fi/~tohkaj/gamixture.html. In addition, the authors
compared their segmentation to that of the software packages Statistical
Parametric Mapping (SPM) and FMRIB Software Library (FSL). The versions of SPM
and FSL used are not specified. SPM is available for download at
http://www.fil.ion.ucl.ac.uk/spm/ and FLS is available at
https://fsl.fmrib.ox.ac.uk/fsl/fslwiki/. Plots include histograms of voxel
values, particularly a set of histograms comparing the ground truth to the
GA-VEM output. The authors also included sample MR images, colored by tissue
classification. These images allow for visual comparison of the ground truth to
the various algorithms used in the experiments. Line graphs indicate the
classification accuracy of each method on the test images. Box and whisker plots
are used to compare the output accuracy for different initial values of
\(\beta^0\). The metric used for comparing the algorithms is classification
accuracy, specifically the Jaccard similarity coefficient. This is the
intersection of the algorithm output with the ground truth, divided by their
union. The ground truth was obtained from investigators trained in brain MR
image segmentation.

The authors claim that using GA overcomes the difficulty of initializing VEM.
Their GA-VEM algorithm results in more accurate classification than the
alternatives compared in the experiments.

The only issue noted with GA-VEM by the authors is determining the initial value
of the \(\beta^0\) hyperparameter. The authors ran a set of experiments with
different initial values of \(\beta^0\), to evaluate the effect on
classification accuracy. The authors made no mention of ideas for future
improvement.

In my opinion, this paper presents an interesting improvement on EM and VEM. The
genetic algorithm seems like a novel method of initializing the VEM algorithm.
This does involve computational complexity than EM, but it appears to be just a
constant factor; that is, given complexity O(n) for EM, the complexity of GA-VEM
appears to be kO(n). The experimental data set seems a bit small, only 37 MR
images. The results are promising, though. GA-VEM is shown to be more accurate
than EM, VEM, and GA-EM. Against SPM and FSL, GA-VEM is more accurate, or only
slightly less accurate.

At this time, I do not have any thoughts on improving this algorithm.

It seems clear that GA-VEM can be applied to any image or video processing tasks
for which EM is used.

em4gmm is a C implementation of the EM algorithm \\
(https://github.com/juandavm/em4gmm). The OpenCV library also contains an EM
implementation \\
(http://docs.opencv.org/2.4/modules/ml/doc/expectation\_maximization.html).

\bibliographystyle{IEEEtran}
\bibliography{references}
\end{document}
