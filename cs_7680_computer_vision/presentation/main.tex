\documentclass[mathserif]{beamer}
%\documentclass[mathserif,handout]{beamer}

%\usepackage{algorithm2e}
\usepackage{mathtools}
\usepackage{booktabs}
\usetheme{metropolis}
%\mode{presentation}

\DeclareMathOperator{\struckDelta}{\Delta \left( \mathbf{y}, \mathbf{y}_i \right)}
\DeclareMathOperator*{\argmax}{arg\,max}
\DeclareMathOperator{\st}{\text{s.t.} \quad}

% front matter
\title{Struck}
\subtitle{Structured Output Tracking with Kernels}
\author{Brendan Robeson}
\date[CS 7680]{CS 7680 - Advanced Computer Vision}
\institute{Utah State University}

\begin{document}
\begin{frame}
    \titlepage
\end{frame}

\begin{frame}
    \frametitle{Outline}
    \tableofcontents
\end{frame}

%\section{Tracking by Detection}
%
%\begin{frame}
%    \huge{Tracking by Detection}
%\end{frame}

\Group{Tracking by Detection}

\begin{frame}
    \frametitle{How do we track an object from frame to frame?}
    \begin{columns}[T]
        \begin{column}{0.5\textwidth}
            \begin{figure}
                \includegraphics[width=0.9\textwidth]{surfer_marked}
                \caption{Initial frame: The surfer location is known.}
            \end{figure}
        \end{column}
        \begin{column}{0.5\textwidth}
            \begin{figure}
                \includegraphics[width=0.9\textwidth]{surfer_unmarked}
                \caption{Subsequent frame: Where is the surfer?}
            \end{figure}
        \end{column}
    \end{columns}
\end{frame}

\begin{frame}
    \frametitle{Tracking can be treated as an object detection problem.}
    \begin{columns}[T]
        \begin{column}{0.5\textwidth}
            \begin{description}
                \item [Tracking by detection] Attempt to detect the object in each frame.
                %\item [Adaptive tracking by detection] Update the classifier online.
            \end{description}
            \begin{itemize}
                \item Transform a tracking problem into an object detection problem.
                \item Can use prior knowledge to assist the detection.
                \item For example, in the previous frame, the object was here. In this frame, it's
                    unlikely to be way over there.
            \end{itemize}
        \end{column}
        \begin{column}{0.5\textwidth}
            \begin{itemize}
                \item The algorithm operates on frame \(f_t\), for \(t \in \{1, 2, ..., T\}\).
                \item Sample \(f_t\) around position \(\vec{p}_{t-1}\).
                \item Extract features \(\vec{x}_i\) for each sample.
                \item Input features to a classifier.
                \item Classifier determines which feature corresponds to the tracked object.
                \item Determine new position \(\vec{p}_t\).
                \item Retrain the classifier with new features. \alert{optional}
            \end{itemize}
        \end{column}
    \end{columns}
\end{frame}

\begin{frame}
    \frametitle{Proposal: Fuzzy Struck}
    \begin{itemize}
        \item Fuzzy SVM incorporates fuzzy logic membership into an SVM. \cite{991432}
        \item Struck uses a structured output SVM. \cite{6126251} \cite{Schwenker2014}
        \item Overall, best performance in a recent survey. \cite{6671560}
        \item Two issues:
            \begin{itemize}
                \item Training samples often contain some object features and some background
                    features.
                \item Target object can change appearance over time, but this is not considered in
                    Struck.
            \end{itemize}
        \item Combine fuzzy SVM with Struck's SVM.
    \end{itemize}
\end{frame}


\section{Struck}

\begin{frame}
    \frametitle{Here's the gist...}

    \begin{itemize}
        \item Don't learn a binary classifier. Learn a prediction function: $f :
            \mathcal{X} \rightarrow \mathcal{Y}$.
        \item Labelled sample is $(\mathbf{x}, \mathbf{y})$.
        \item $f$ is learned in a structured output SVM.
        \item Use a discriminant function $F: \mathcal{X} \times \mathcal{Y}
            \rightarrow \mathbb{R}$.
        \item Prediction function becomes %\uncover<1>{
            \begin{align}
                \mathbf{y}_t &= \argmax\limits_{y \in \mathcal{Y}} h \left( \mathbf{x}_t^{\mathbf{p}_{t-1}}, \mathbf{y} \right) \tag{prior} \\
                \mathbf{y}_t &= \argmax\limits_{y \in \mathcal{Y}} F \left( \mathbf{x}_t^{\mathbf{p}_{t-1}}, \mathbf{y} \right) \tag{Struck}
            \end{align}
    \end{itemize}
\end{frame}

\begin{frame}
    \frametitle{What is the discriminant function, $F$?}
    \begin{equation}
        F(\mathbf{x}, \mathbf{y}) = \langle \mathbf{w}, \mathbf{\Phi}(\mathbf{x}, \mathbf{y}\rangle
    \end{equation}
    \begin{equation} \label{eq:minimizing_hyperplane}
        \begin{aligned}
            \min_{\mathbf{w}} \quad & \frac{1}{2} \|\mathbf{w}\|^2 + C \sum_{i=1}^n \xi_i \\
            \st & \forall i : \xi \ge 0 \\
                & \forall i, \forall \mathbf{y} \ne \mathbf{y}_i : \langle \mathbf{w}, \delta \mathbf{\Phi}_i (\mathbf{y}) \rangle \ge \Delta (\mathbf{y}_i, \mathbf{y}) - \xi_i
        \end{aligned}
    \end{equation}

    \begin{equation}
        \delta \mathbf{\Phi}_i (\mathbf{y}) = \mathbf{\Phi}(\mathbf{x}_i, \mathbf{y}_i) - \mathbf{\Phi}(\mathbf{x}_i, \mathbf{y})
    \end{equation}
\end{frame}

\begin{frame}
    \frametitle{What is the discriminant function, $F$}
    \begin{itemize}
        \item $\Delta$ is a loss function.
            \begin{itemize}
                \item $\Delta (\mathbf{y}, \mathbf{\bar{y}}) = 0 \; \text{iff} \; \mathbf{y} = \mathbf{\bar{y}}$
                \item $\Delta$ should decrease to 0 as $\mathbf{y}$ approaches $\mathbf{\bar{y}}$
            \end{itemize}
        \item This overcomes the equal treatment of samples.
        \item The authors use a variation of bounding box overlap:
    \end{itemize}
    \begin{equation}
        \Delta(\mathbf{y}, \mathbf{\bar{y}}) = 1 - s_{\mathbf{p}_t}^o (\mathbf{y}, \mathbf{\bar{y}})
    \end{equation}
    \alert{What does this mean for $\Delta$?}
\end{frame}

\begin{frame}
    \frametitle{Optimizing equation \eqref{eq:minimizing_hyperplane}}
    \begin{equation}
        \min_{\mathbf{w}} \frac{1}{2} \|\mathbf{w}\|^2 + C \sum_{i=1}^n \xi_i \tag{\ref{eq:minimizing_hyperplane}}
    \end{equation}
    \begin{equation}
        \begin{aligned}
            \max_{\alpha} \quad & \sum_{i, \mathbf{y} \ne \mathbf{y}_i} \struckDelta \alpha_i^\mathbf{y} -
                \frac{1}{2} \sum_{\substack{i, \mathbf{y} \ne \mathbf{y}_i \\ j, \mathbf{y} \ne \mathbf{y}_j}}
                \alpha_i^\mathbf{y} \alpha_j^\mathbf{\bar{y}} \langle \delta \mathbf{\Phi}_i \left( \mathbf{\bar{y}} \right),
                \delta \mathbf{\Phi}_j \left( \mathbf{\bar{y}} \right) \rangle \\
            \st & \forall i, \forall \mathbf{y} \ne \mathbf{y}_i : \alpha_i^\mathbf{y} \ge 0 \\
                & \forall i \sum_{\mathbf{y} \ne \mathbf{y}_i} \alpha_i^\mathbf{y} \le C
        \end{aligned}
    \end{equation}
\end{frame}

\begin{frame}
    \frametitle{Optimizing equation \eqref{eq:minimizing_hyperplane}}
    Reparameterising based on ...
    \begin{equation}
        \beta_i^\mathbf{y} = \begin{dcases}
                                -\alpha_i^\mathbf{y} & \mathbf{y} \ne \mathbf{y}_i \\
                                \sum_{\mathbf{\bar{y}} \ne \mathbf{y}_i} \alpha_i^\mathbf{\bar{y}} & \mathbf{y} = \mathbf{y}_i
                             \end{dcases}
    \end{equation}

    \begin{equation}
        \begin{aligned}
            \max_{\beta} & -\sum_{i,\mathbf{y}} \struckDelta \beta_i^\mathbf{y} -
                \frac{1}{2} \sum_{i,\mathbf{y},j,\mathbf{\bar{y}}} \beta_i^\mathbf{y}
                \beta_j^\mathbf{\bar{y}} \langle \mathbf{\Phi}\left( \mathbf{x}_i, \mathbf{y}
                \right), \mathbf{\Phi}\left( \mathbf{x}_j, \mathbf{\bar{y}} \right) \rangle \\
            \st & \forall i, \forall \mathbf{y} : \beta_i^\mathbf{y} \le \delta(\mathbf{y}, \mathbf{y}_i) C \\
                & \forall i: \sum_\mathbf{y} \beta_i^\mathbf{y} = 0
        \end{aligned}
    \end{equation}
\end{frame}
\begin{frame}
    \frametitle{Optimizing equation \eqref{eq:minimizing_hyperplane}}
    \begin{gather}
        \delta(\mathbf{y}, \mathbf{\bar{y}}) = \begin{cases}
                                                    1 & \mathbf{y} = \mathbf{\bar{y}} \\
                                                    0 & \mathbf{y} \ne \mathbf{\bar{y}}
                                               \end{cases} \\
        F(\mathbf{x}, \mathbf{y}) = \sum_{i, \mathbf{\bar{y}}} \beta_i^\mathbf{\bar{y}} \langle
            \mathbf{\Phi}(\mathbf{x}_i, \mathbf{\bar{y}}), \mathbf{\Phi}(\mathbf{x}, \mathbf{y}) \rangle
    \end{gather}
\end{frame}

%\begin{frame}
%    \frametitle{How does Struck work?}
%    \begin{itemize}
%        \item Learning and tracking are integrated.
%        \item Uses a structured output SVM.
%        \item Must overcome the \alert{curse of kernelization.}
%            \begin{itemize}
%                \item Number of support vectors increase as a function of training data
%            \end{itemize}
%    \end{itemize}
%\end{frame}

\section{Experiments}

\begin{frame}
    \frametitle{Data sets}
    \url{http://vision.ucsd.edu/~bbabenko/project_miltrack.html}
    \begin{itemize}
        \item 12 sequences
        \item 8 used in \cite{6126251}.
    \end{itemize}
    \url{http://alov300.org/}
    \begin{itemize}
        \item 314 sequences used in \cite{6671560}.
        \item Categorized to test various object tracking challenges.
    \end{itemize}
\end{frame}

\begin{frame}
    \frametitle{Experimental Setup}
    \begin{itemize}
        \item Set up as described in their paper:
            \begin{itemize}
                \item Haar features
                \item Gaussian kernel \(\sigma\) = 0.2
                \item \(C \in [1, 100]\)
                \item Search radius \(r\) = 30 pixels
                \item SVM budget = 100
                \item Frame size = 320 x 240
                \item Random seed = 0
            \end{itemize}
    \end{itemize}
\end{frame}

\begin{frame}
    \frametitle{Scripting the Experiments}
    \begin{algorithm}[H]
        \DontPrintSemicolon
        \KwIn{Sequences}
        \Begin(Run experiments)
        {
            \ForEach{\(C \in [1, 100]\)}
            {
                \ForEach{Sequence}
                {
                    Run Struck\;
                    Compare output with ground truth\;
                    Record mean IoU\;
                }
            }
        }
        \KwOut{IoU results}
    \end{algorithm}
\end{frame}

\begin{frame}
    \frametitle{Results}
    \begin{center}
        %\begin{center}
%\begin{tabular}{l r r r}
%    \toprule
%    Sequence & Struck & \(s_{min}\) removal & no \(s_i = 0\) \\
%    \midrule
%    girl     &         0.681729  & \textbf{0.681821} & \textbf{0.681821} \\
%    surfer   &         0.787232  &         0.787232  & \textbf{0.811688} \\
%    tiger1   &         0.674713  &         0.651670  & \textbf{0.746060} \\
%    tiger2   &         0.667268  &         0.509582  & \textbf{0.679176} \\
%    twinings &         0.903097  & \textbf{0.931334} & \textbf{0.907482} \\
%    dollar   &         0.765203  &         0.765203  &         0.765203  \\
%    faceocc  &         0.853354  &         0.853354  &         0.853354  \\
%    cliffbar & \textbf{0.894553} &         0.890760  &         0.717940  \\
%    coke11   & \textbf{0.883395} &         0.882497  &         0.841359  \\
%    david    & \textbf{0.800847} &         0.434802  &         0.434802  \\
%    faceocc2 & \textbf{0.790128} &         0.789270  &         0.789270  \\
%    sylv     & \textbf{0.861525} &         0.857138  &         0.858211  \\
%    \bottomrule
%\end{tabular}

%\begin{tabular}{l r r}
%    \toprule
%    Sequence & Struck & Fuzzy Struck \\
%    \midrule
%    girl     &         0.681729  & \textbf{0.681821} \\
%    surfer   &         0.787232  & \textbf{0.811688} \\
%    tiger1   &         0.674713  & \textbf{0.746060} \\
%    tiger2   &         0.667268  & \textbf{0.679176} \\
%    twinings &         0.903097  & \textbf{0.907482} \\
%    dollar   &         0.765203  &         0.765203  \\
%    faceocc  &         0.853354  &         0.853354  \\
%    cliffbar & \textbf{0.894553} &         0.717940  \\
%    coke11   & \textbf{0.883395} &         0.841359  \\
%    david    & \textbf{0.800847} &         0.434802  \\
%    faceocc2 & \textbf{0.790128} &         0.789270  \\
%    sylv     & \textbf{0.861525} &         0.858211  \\
%    \bottomrule
%\end{tabular}

% haar
%\begin{tabular}{l r r}
%    \toprule
%    Sequence & Struck & Fuzzy Struck \\
%    \midrule
%    surfer   &         0.78  & \textbf{0.81} \\
%    tiger1   &         0.67  & \textbf{0.74} \\
%    tiger2   &         0.66  & \textbf{0.67} \\
%    girl     &         0.68  &         0.68  \\
%    twinings &         0.90  &         0.90  \\
%    dollar   &         0.76  &         0.76  \\
%    faceocc  &         0.85  &         0.85  \\
%    cliffbar & \textbf{0.89} &         0.71  \\
%    coke11   & \textbf{0.88} &         0.84  \\
%    david    & \textbf{0.80} &         0.43  \\
%    faceocc2 & \textbf{0.79} &         0.78  \\
%    sylv     & \textbf{0.86} &         0.85  \\
%    \bottomrule
%\end{tabular}


% histogram
\begin{table}[!t]
    \caption{Averange Bounding Box Overlap}
    \label{tab:results}
    \centering
    \begin{tabular}{l r r}
        \toprule
        Sequence & Fuzzy Struck & Struck \\
        \midrule
        cliffbar &  \textbf{0.94} &         0.91  \\
        coke11   &  \textbf{0.94} &         0.93  \\
        faceocc  &  \textbf{0.89} &         0.88  \\
        sylv     &  \textbf{0.89} &         0.74  \\
        tiger1   &  \textbf{0.87} &         0.82  \\
        tiger2   &  \textbf{0.80} &         0.66  \\
        twinings &  \textbf{0.92} &         0.87  \\
        surfer   &          0.86  &         0.86  \\
        david    &          0.81  & \textbf{0.83} \\
        dollar   &          0.89  & \textbf{0.90} \\
        faceocc2 &          0.73  & \textbf{0.76} \\
        girl     &          0.54  & \textbf{0.60} \\
        \bottomrule
    \end{tabular}
\end{table}
%\end{center}

        %\begin{tabular}{l c c c }
        %    \toprule
        %    Video & Struck\textsubscript{\cite{6126251}} & Struck\textsubscript{Mine} & Weighted Struck \\
        %    \midrule
        %    Coke      &         0.57  & \textbf{0.88} &         0.84  \\
        %    David     &         0.80  & \textbf{0.86} &         0.48  \\
        %    Face 1    & \textbf{0.86} & \textbf{0.86} &         0.81  \\
        %    Face 2    & \textbf{0.86} &         0.80  &         0.80  \\
        %    Girl      & \textbf{0.80} &         0.68  &         0.67  \\
        %    Sylvester &         0.68  & \textbf{0.86} & \textbf{0.86} \\
        %    Tiger 1   &         0.70  &         0.67  & \textbf{0.74} \\
        %    Tiger 2   &         0.57  &         0.67  & \textbf{0.68} \\
        %    \bottomrule
        %\end{tabular}
    \end{center}
\end{frame}

%\begin{frame}
%    \frametitle{Continous IoU}
%    \begin{center}
%        \includegraphics[height=0.8\textheight]{ious}
%    \end{center}
%\end{frame}
%
%\begin{frame}
%    \frametitle{Sample frame}
%    \begin{center}
%        \includegraphics[height=0.9\textheight]{tiger1_210}
%    \end{center}
%\end{frame}


\section{Conclusion}

\begin{frame}
    \frametitle{What does Struck do?}
    Struck provides an adaptive tracking-by-detection framework.

    It directly ties image features to a transformation.

    It eliminates binary labelling with equal weight.

    It uses a structured output SVM, with a boundary on the number of support vectors.
\end{frame}

\begin{frame}
    \frametitle{Where does Struck fall short?}
    Currently, Struck focuses solely on translation.

    Though rotation does not appear to confuse Struck, it's not detected and shown.

    Smeulders et. al. concluded Struck's primary problem was dealing with scale (camera zoom).
\end{frame}

\begin{frame}
    \frametitle{References}
    \nocite{*}
    \bibliographystyle{IEEEtran}
    \bibliography{references}

    Source code: \url{https://github.com/samhare/struck}
\end{frame}


\end{document}

