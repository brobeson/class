\documentclass[mathserif,handout]{beamer}
\usepackage{algorithm2e}
%\documentclass{beamer}
\usepackage{booktabs}
\usetheme{metropolis}
%\mode{presentation}

\DeclareMathOperator*{\argmax}{arg\,max}

% front matter
\title{Struck}
\subtitle{Structured Output Tracking with Kernels}
\author{Brendan Robeson}
\date[CS 7680]{CS 7680 - Advanced Computer Vision}
\institute{Utah State University}

\begin{document}
\begin{frame}
    \titlepage
\end{frame}

\begin{frame}
    \frametitle{Outline}
    \tableofcontents
\end{frame}

\section{The Problem}

\begin{frame}
    \frametitle{Tracking can be treated as an object detection problem.}
    \begin{description}
        \item [Tracking by detection] Object detection in each frame.
        \item [Adaptive tracking by detection] Update the classifier online.
    \end{description}

    Separates sample generation from classifier updating.
\end{frame}

%\begin{frame}
%    \frametitle{So, what's wrong with the previous algorithms?}
%    \begin{enumerate}
%        \item<1-> How should samples be generated and labeled?
%
%            \uncover<2->{\alert{Most consider samples far from the target to be negative.}}
%        \item<1-> Classifier wants to determine \emph{what} an object is.
%
%            Tracker wants to determine \emph{where} an object is.
%
%            \uncover<3->{\alert{The two are not trained together. Thus, assuming highest classifier
%            output $\Rightarrow$ location is not always true.}}
%    \end{enumerate}
%    \uncover<4->{Most trackers try to make the classifier more robust with respect to poor samples.}
%\end{frame}

%\begin{frame}
%    \frametitle{Adaptive tracking-by-detection algorithm}
%    \begin{algorithm}[H]
%        \DontPrintSemicolon
%        \KwIn{Video frame: $f_t$}
%            \ForEach{ROI image}
%            {
%                Calculate the GLCM $P_u$\;
%                Calculate energy using Equation (2)\;
%                Calculate contrast using Equation (3)\;
%                Calculate correlation using Equation (4)\;
%                Calculate entropy using Equation (5)\;
%                Calculate homogeneity using Equation (6)\;
%            }
%            \KwOut{A set of feature vectors}
%    \end{algorithm}
%\end{frame}

\begin{frame}
    \frametitle{How does one train adaptive tracking-by-detection?}
    The algorithm operates on frame $f_t$, for $t \in \{1, 2, ..., T\}$.

    The tracker estimates a bounding box position, $\mathbf{p}$.

    Extract features $\mathbf{x}_t^\mathbf{p}$ from patches within the bounding box.

    Train the classifier with $(\mathbf{x}, z)$\uncover<3->{, where $z = \pm1$.}

    \uncover<2>{\alert{What is $z$?}}
\end{frame}

\begin{frame}
    \frametitle{How does one predict using adaptive tracking-by-detection?}
    Uses a classification confidence function $h(\mathbf{x})$

    Goal is to estimate a transformation $\mathbf{y}_t \in \mathcal{Y}$, where $\mathbf{p}_t =
    \mathbf{p}_{t-1} \circ \mathbf{y}_t$

    $\mathcal{Y}$ is the search space

    Typically, $\mathcal{Y} = \left\{ (u,v) | u^2 + v^2 < r^2 \right \}$ \alert{$\Leftarrow$ What does
    this mean?}

    $\mathbf{y}_t = \argmax\limits_{y \in \mathcal{Y}} h \left( \mathbf{x}_t^{\mathbf{p}_{t-1}},
    \mathbf{y} \right)$
\end{frame}

\begin{frame}
    \frametitle{Adapting the classifier.}
    Update the classifier after calculating $\mathbf{p}_t$

    Generate sample transformations: $\left\{ \mathbf{y}_t^1, \mathbf{y}_t^2, ..., \mathbf{y}_t^n
    \right\}$

    Corresponding training samples: $\left\{ \mathbf{x}_t^{\mathbf{p}_t \circ \mathbf{y}_t^1},
    \mathbf{x}_t^{\mathbf{p}_t \circ \mathbf{y}_t^2}, ..., \mathbf{x}_t^{\mathbf{p}_t \circ
    \mathbf{y}_t^n} \right\}$

    Corresponding labels: $\left\{ z_t^1, z_t^2, ..., z_t^n \right\}$

    Use the samples and labels to update the classifier.
\end{frame}

%\section{Prior Solutions}

\section{Struck}

\begin{frame}
    \frametitle{How does Struck work?}
    \begin{itemize}
        \item Learning and tracking are integrated.
        \item Uses a structured output SVM.
        \item Must overcome the \alert{curse of kernelization.}
            \begin{itemize}
                \item Number of support vectors increase as a function of training data
            \end{itemize}
    \end{itemize}
\end{frame}

\section{Experiments}

\begin{frame}
    \frametitle{How did they measure performance?}
    \begin{block}{Bounding box overlap}
        $\text{IoU} = \frac{B \cap G}{B \cup G}$
    \end{block}
\end{frame}

\begin{frame}
    \frametitle{Bounding box overlap}
    \begin{tabular}{l c c c c }
        \toprule
        Video & Struck\textsubscript{$\infty$} & Struck\textsubscript{100} & Struck\textsubscript{50} & Struct\textsubscript{20} \\
        \midrule
        Coke      & \textbf{0.57} & \textbf{0.57} &         0.56  & \underline{0.52} \\
        David     &         0.80  &         0.80  & \textbf{0.81} &            0.35  \\
        Face 1    &         0.86  &         0.86  &         0.86  &            0.81  \\
        Face 2    & \textbf{0.86} & \textbf{0.86} & \textbf{0.86} & \underline{0.83} \\
        Girl      & \textbf{0.80} & \textbf{0.80} & \textbf{0.80} & \underline{0.79} \\
        Sylvester & \textbf{0.68} & \textbf{0.68} &         0.67  &            0.58  \\
        Tiger 1   & \textbf{0.70} & \textbf{0.70} &         0.69  & \underline{0.68} \\
        Tiger 2   &         0.56  & \textbf{0.57} &         0.55  &            0.39  \\
        \bottomrule
    \end{tabular}
\end{frame}

\begin{frame}
    \frametitle{Bounding box overlap}
    \begin{tabular}{l c c c c c c }
        \toprule
        Video & Struck & MIForest & OMCLP & MIL & Frag & OAB \\
        \midrule
        Coke      & \textbf{0.57} & 0.35 & 0.24 & 0.33 &         0.08  & 0.17 \\
        David     & \textbf{0.81} & 0.72 & 0.61 & 0.57 &         0.43  & 0.26 \\
        Face 1    &         0.86  & 0.77 & 0.80 & 0.60 & \textbf{0.88} & 0.48 \\
        Face 2    & \textbf{0.86} & 0.77 & 0.78 & 0.68 &         0.44  & 0.68 \\
        Girl      & \textbf{0.80} & 0.71 & 0.64 & 0.53 &         0.60  & 0.40 \\
        Sylvester & \textbf{0.68} & 0.59 & 0.67 & 0.60 &         0.62  & 0.52 \\
        Tiger 1   & \textbf{0.70} & 0.55 & 0.53 & 0.52 &         0.19  & 0.23 \\
        Tiger 2   & \textbf{0.57} & 0.53 & 0.44 & 0.53 &         0.15  & 0.28 \\
        \bottomrule
    \end{tabular}

    The Struck column shows the best values from the previous table.
\end{frame}

\begin{frame}
    \frametitle{Processing time}
    \begin{tabular}{l c c c c }
        \toprule
        Video & Struck\textsubscript{$\infty$} & Struck\textsubscript{100} & Struck\textsubscript{50} & Struct\textsubscript{20} \\
        \midrule
        Average FPS & 12.1    & 13.2    & 16.2    & 21.4 \\
        %Average SPF &  0.0826 &  0.0758 &  0.0617 &  0.0467 \\
        \bottomrule
    \end{tabular}

    Not keeping up with typical U.S. frame rate!
\end{frame}

%\begin{frame}
%    \frametitle{Combining kernels}
%    \begin{tabular}{l c c c c c c }
%        \toprule
%        Video & Struck & MIForest & OMCLP & MIL & Frag & OAB \\
%        \midrule
%        Coke      & \textbf{0.57} & 0.35 & 0.24 & 0.33 &         0.08  & 0.17 \\
%        David     & \textbf{0.81} & 0.72 & 0.61 & 0.57 &         0.43  & 0.26 \\
%        Face 1    &         0.86  & 0.77 & 0.80 & 0.60 & \textbf{0.88} & 0.48 \\
%        Face 2    & \textbf{0.86} & 0.77 & 0.78 & 0.68 &         0.44  & 0.68 \\
%        Girl      & \textbf{0.80} & 0.71 & 0.64 & 0.53 &         0.60  & 0.40 \\
%        Sylvester & \textbf{0.68} & 0.59 & 0.67 & 0.60 &         0.62  & 0.52 \\
%        Tiger 1   & \textbf{0.70} & 0.55 & 0.53 & 0.52 &         0.19  & 0.23 \\
%        Tiger 2   & \textbf{0.57} & 0.53 & 0.44 & 0.53 &         0.15  & 0.28 \\
%        \bottomrule
%    \end{tabular}
%\end{frame}

\section{Conclusion}

\begin{frame}
    \frametitle{References}
    \nocite{*}
    \bibliographystyle{IEEEtran}
    \bibliography{references}

    Source code: \url{https://github.com/samhare/struck}
\end{frame}

\end{document}

