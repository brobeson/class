\section{Struck}

\begin{frame}
    \frametitle{Here's the gist...}
    \begin{itemize}
        \item Don't learn a binary classifier. Learn a prediction function: \(f :
            \mathcal{X} \rightarrow \mathcal{Y}\).
        \item Labelled sample is \((\mathbf{x}, \mathbf{y})\).
        \item \(f\) is learned in a structured output SVM.
        \item Use a discriminant function \(F: \mathcal{X} \times \mathcal{Y}
            \rightarrow \mathbb{R}\).
        \item Prediction function becomes %\uncover<1>{
            \begin{align}
                \mathbf{y}_t &= \argmax\limits_{y \in \mathcal{Y}} h \left( \mathbf{x}_t^{\mathbf{p}_{t-1}}, \mathbf{y} \right) \tag{prior} \\
                \mathbf{y}_t &= \argmax\limits_{y \in \mathcal{Y}} F \left( \mathbf{x}_t^{\mathbf{p}_{t-1}}, \mathbf{y} \right) \tag{Struck}
            \end{align}
    \end{itemize}
\end{frame}

\begin{frame}
    \frametitle{What is the discriminant function, \(F\)?}
    \begin{equation}
        F(\mathbf{x}, \mathbf{y}) = \langle \mathbf{w}, \mathbf{\Phi}(\mathbf{x}, \mathbf{y})\rangle
    \end{equation}

    \begin{equation} \label{eq:minimizing_hyperplane}
        \begin{aligned}
            \min_{\mathbf{w}} \quad & \frac{1}{2} \|\mathbf{w}\|^2 + C \sum_{i=1}^n \xi_i \\
            \st & \forall i : \xi \ge 0 \\
                & \forall i, \forall \mathbf{y} \ne \mathbf{y}_i : \langle \mathbf{w}, \delta \mathbf{\Phi}_i (\mathbf{y}) \rangle \ge \Delta (\mathbf{y}_i, \mathbf{y}) - \xi_i
        \end{aligned}
    \end{equation}

    where \( \delta \mathbf{\Phi}_i (\mathbf{y}) = \mathbf{\Phi}(\mathbf{x}_i, \mathbf{y}_i) - \mathbf{\Phi}(\mathbf{x}_i, \mathbf{y}) \)
\end{frame}

\begin{frame}
    \frametitle{What is the discriminant function, \(F\)}
    \begin{itemize}
        \item \(\Delta\) is a loss function.
            \begin{itemize}
                \item \(\Delta (\mathbf{y}, \mathbf{\bar{y}}) = 0 \; \text{iff} \; \mathbf{y} = \mathbf{\bar{y}}\)
                \item \(\Delta\) should decrease to 0 as \(\mathbf{y}\) approaches \(\mathbf{\bar{y}}\)
            \end{itemize}
        \item This overcomes the equal treatment of samples.
        \item The authors use a variation of bounding box overlap:
    \end{itemize}
    \begin{equation}
        \Delta(\mathbf{y}, \mathbf{\bar{y}}) = 1 - s_{\mathbf{p}_t}^o (\mathbf{y}, \mathbf{\bar{y}})
    \end{equation}
    \alert{What does this mean for \(\Delta\)?}
\end{frame}

\begin{frame}
    \frametitle{Optimizing equation \eqref{eq:minimizing_hyperplane}}
    \begin{equation}
        \min_{\mathbf{w}} \frac{1}{2} \|\mathbf{w}\|^2 + C \sum_{i=1}^n \xi_i \tag{\ref{eq:minimizing_hyperplane}}
    \end{equation}

    \begin{equation}
        \begin{aligned}
            \max_{\alpha} \quad & \sum_{i, \mathbf{y} \ne \mathbf{y}_i} \struckDelta \alpha_i^\mathbf{y} -
                \frac{1}{2} \sum_{\substack{i, \mathbf{y} \ne \mathbf{y}_i \\ j, \mathbf{y} \ne \mathbf{y}_j}}
                \alpha_i^\mathbf{y} \alpha_j^\mathbf{\bar{y}} \langle \delta \mathbf{\Phi}_i \left( \mathbf{\bar{y}} \right),
                \delta \mathbf{\Phi}_j \left( \mathbf{\bar{y}} \right) \rangle \\
            \st & \forall i, \forall \mathbf{y} \ne \mathbf{y}_i : \alpha_i^\mathbf{y} \ge 0 \\
                & \forall i \sum_{\mathbf{y} \ne \mathbf{y}_i} \alpha_i^\mathbf{y} \le C
        \end{aligned}
    \end{equation}

    via Lagrangian duality
\end{frame}

\begin{frame}
    \frametitle{Optimizing equation \eqref{eq:minimizing_hyperplane}}
    Reparameterising based on ...
    \begin{equation}
        \beta_i^\mathbf{y} = \begin{dcases}
                                -\alpha_i^\mathbf{y} & \mathbf{y} \ne \mathbf{y}_i \\
                                \sum_{\mathbf{\bar{y}} \ne \mathbf{y}_i} \alpha_i^\mathbf{\bar{y}} & \mathbf{y} = \mathbf{y}_i
                             \end{dcases}
    \end{equation}

    \begin{equation}
        \begin{aligned}
            \max_{\beta} \quad & -\sum_{i,\mathbf{y}} \struckDelta \beta_i^\mathbf{y} -
                \frac{1}{2} \sum_{i,\mathbf{y},j,\mathbf{\bar{y}}} \beta_i^\mathbf{y}
                \beta_j^\mathbf{\bar{y}} \langle \mathbf{\Phi}\left( \mathbf{x}_i, \mathbf{y}
                \right), \mathbf{\Phi}\left( \mathbf{x}_j, \mathbf{\bar{y}} \right) \rangle \\
            \st & \forall i, \forall \mathbf{y} : \beta_i^\mathbf{y} \le \delta(\mathbf{y}, \mathbf{y}_i) C \\
                & \forall i: \sum_\mathbf{y} \beta_i^\mathbf{y} = 0 \\
                & \delta(\mathbf{y}, \mathbf{\bar{y}}) = \begin{cases}
                                                    1 & \mathbf{y} = \mathbf{\bar{y}} \\
                                                    0 & \mathbf{y} \ne \mathbf{\bar{y}}
                                               \end{cases} \nonumber \\
        \end{aligned}
    \end{equation}
\end{frame}

%\begin{frame}
%    \frametitle{Optimizing equation \eqref{eq:minimizing_hyperplane}}
%    \begin{align}
%        \delta(\mathbf{y}, \mathbf{\bar{y}}) &= \begin{cases}
%                                                    1 & \mathbf{y} = \mathbf{\bar{y}} \\
%                                                    0 & \mathbf{y} \ne \mathbf{\bar{y}}
%                                               \end{cases} \nonumber \\
%        F(\mathbf{x}, \mathbf{y}) &= \sum_{i, \mathbf{\bar{y}}} \beta_i^\mathbf{\bar{y}} \langle
%            \mathbf{\Phi}(\mathbf{x}_i, \mathbf{\bar{y}}), \mathbf{\Phi}(\mathbf{x}, \mathbf{y}) \rangle
%    \end{align}
%\end{frame}

%\begin{frame}
%    \frametitle{How does Struck work?}
%    \begin{itemize}
%        \item Learning and tracking are integrated.
%        \item Uses a structured output SVM.
%        \item Must overcome the \alert{curse of kernelization.}
%            \begin{itemize}
%                \item Number of support vectors increase as a function of training data
%            \end{itemize}
%    \end{itemize}
%\end{frame}
