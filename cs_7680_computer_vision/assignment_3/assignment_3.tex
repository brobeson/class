\documentclass[11pt]{article}
\usepackage[margin=1in]{geometry}
\usepackage{enumitem}
\usepackage{algorithm2e}
\usepackage{hyperref}
\hypersetup{colorlinks=true}

\begin{document}
\noindent Brendan Robeson

\noindent CS 7680 - Assignment 3

\noindent \today

\medskip

\begin{description}[leftmargin=0in]
    \item [Source] G. ming Xian, "An identification method of malignant and
        benign liver tumors from ultrasonogrpahy based on GLCM texture features
        and fuzzy SVM," \emph{Expert Systems with Applications}, vol. 37, no.
        10, pp. 6737-6741, 2010. [Online]. Cited by 76.

    \item [URL]
        \url{http://www.sciencedirect.com/science/article/pii/S0957417410001065}

    \item [Problem] The author wants to train a computer to correctly classify
        liver tumors as malignant or benign. Improving a computer's accuracy in
        such a task has clear benefits for both doctors and patients.

    \item [Assumptions] The author does not state any assumptions.

    \item [Data Sets] The author used two data sets in his experiments. The
        first consisted of 200 images. 132 were known to be benign, and 68 were
        known to be malignant. The second data set had 450 images, all with
        various benign tumors. No URLs are provided for the data sets. 

    \item [Algorithm Overview]

    \item [Experiments] The author only compared their technique to standard
        SVM. The only comparison data presented are a table of misdiagnoses by
        each algorithm.

        Other tables and graphs are provided with data relevant only to the
        proposed technique. Among these are a graph of the technique's accuracy
        as a function of the GLCM's $\delta$ value. Another table presents
        FSVM's performance for five metrics: accuracy, sensitivity, specificity,
        positive predictive value, and negative predictive value.

        No qualitative figures are provided.

    \item [Contributions] This paper seems to have improved the accuracy of
        automated diagnosis of liver tumors.

    \item [Shortcomings] The author does not mention any shortcomings.

    \item [Self Evaluation] This paper does not appear to develop any new
        techniques, but rather to simply apply FSVM to the application of tumor
        diagnosis. In addition, a physician is still required to identify a
        tumor as a region of interest for input to the algorithm. This technique
        is not much more complex than non-linear kernel soft margin SVMs. The
        results are interesting, but I think comparison with more classification
        techniques would improve the experiments.

    \item [Improvements] GLCM and FSVM are subject to input parameters, the
        values of which influence the quality of the results. A method for
        automatically determining the optimal parameters would improve the
        algorithm. In the case of GLCM, a different feature extraction technique
        could eliminate the $\delta$ parameter.

    \item [Applications] This technique is limited to applications for which
        texture analysis is used for feature extraction, which are then used for
        classification. The FSVM component, though, can be used for any
        classification problem for which fuzzy logic is applicable.

    \item [Packages] \url{https://www.csie.ntu.edu.tw/~cjlin/libsvm/} \\
        \url{https://www.mathworks.com/help/stats/support-vector-machine-classification.html}
\end{description}

 \end{document}
